% !TEX program = pdflatex
% Thermodynamics&StatisticalMechanicsCheatSheetforFinal
\documentclass[10pt,a4paper]{article}
\usepackage{geometry}
\geometry{left=.1cm,right=.1cm,top=.1cm,bottom=.1cm}
\usepackage{multicol}
\setlength{\columnseprule}{1pt}
\usepackage[UTF8,linespread=.1]{ctex}
\usepackage{amsmath,amssymb,mathrsfs,bm,graphicx}
%\usepackage{tipa}
\usepackage{mathrsfs}
\begin{document}
\tiny
\begin{multicols}{2}
\noindent\textbf{Chap7 玻尔兹曼统计}
\textbf{7.1 热力学量的统计表达式}
\textbf{玻尔兹曼分布}:$a_l=\omega_le^{-\alpha-\beta\varepsilon_l}$;
\textbf{配分函数}:$Z_1=\sum_l\omega_le^{-\beta\varepsilon_l}$\\
\textbf{总粒子数}:$N=\sum_la_l=\sum_l\omega_le^{-\alpha-\beta\varepsilon_l}=e^{-\alpha}Z_1$\\
\textbf{内能}:$U=\sum_l\varepsilon_la_l=\sum_l\omega_l\varepsilon_le^{-\alpha-\beta\varepsilon_l}=e^{-\alpha}\left(-\frac{\partial}{\partial\beta}\sum_l\omega_le^{-\beta\varepsilon_l}\right)=-\frac{N}{Z_1}\frac{\partial Z_1}{\partial\beta}=-N\frac{\partial}{\partial\beta}\ln Z_1$\\
\textbf{广义力}:$Y=\sum_l\frac{\partial\varepsilon_l}{\partial y}a_l=\sum_l\frac{\partial\varepsilon_l}{\partial y}\omega_le^{-\alpha-\beta\varepsilon_l}=e^{-\alpha}\left(-\frac{1}{\beta}\frac{\partial}{\partial y}\right)\sum_l\omega_le^{-\beta\varepsilon_l}=-\frac{N}{Z_1}\frac{1}{\beta}\frac{\partial Z_1}{\partial y}=-\frac{N}{\beta}\frac{\partial}{\partial y}\ln Z_1$,其中$y$--广义坐标;\textbf{压强}:$p=\frac{N}{\beta}\frac{\partial}{\partial V}\ln Z_1$\\
\textbf{广义功}:$dW=Ydy=dy\sum_l\frac{\partial\varepsilon_l}{\partial y}a_l=\sum_la_ld\varepsilon_l$,\textbf{内能全微分}:$dU=\sum_l\varepsilon_lda_l+\sum_la_ld\varepsilon_l$,其中第一项为能级不变时粒子分布改变引起的内能变化,第二项为粒子分布不变时能级改变引起的内能变化,比较$dU=dQ+dW$得\textbf{吸热}:$dQ=\sum\varepsilon_lda_l$,这说明系统吸热等于前者,外界对系统做功等于后者\\
\textbf{参数}:$\bm{\beta}$:$dQ=dU-Ydy=-Nd\left(\frac{\partial}{\partial\beta}\ln Z_1\right)+\frac{N}{\beta}\frac{\partial\ln Z_1}{\partial y}dy$,两边同乘$\beta$得$\beta dQ=-N\beta d\left(\frac{\partial}{\partial\beta}\ln Z_1\right)+N\frac{\partial\ln Z_1}{\partial y}dy$,因$Z_1$为$\beta,y$的函数,$d(\ln Z_1)=\frac{\partial\ln Z_1}{\partial\beta}d\beta+\frac{\partial\ln Z_1}{\partial y}dy$,又$d\left(N\beta\frac{\partial\ln Z_1}{\partial\beta}\right)=Nd\left(\beta\frac{\partial\ln Z_1}{\partial\beta}\right)=N\beta d\left(\frac{\partial\ln Z_1}{\partial\beta}\right)+N\frac{\partial\ln Z_1}{\partial\beta}d\beta$,代入前前式得$\beta dQ=d\left(N\ln Z_1-N\beta\frac{\partial\ln Z_1}{\partial\beta}\right)$,这说明$\beta$为一积分因子,熵式$dS=dQ/T$说明$1/T$亦一积分因子,设两者相差一常数$k$,称\textbf{玻尔兹曼常数},$\beta=1/kT$,应用于理想气体得$k=R/N_A=1.381\times10^{-23}J\cdot K^{-1}$;
\textbf{熵}:$S=Nk\left(\ln Z_1-\beta\frac{\partial}{\partial\beta}\ln Z_1\right)$\\
\textbf{玻尔兹曼关系}:将$N=e^{-\alpha}Z_1\Rightarrow\ln Z_1=\ln N+\alpha$和$U=-N\frac{\partial}{\partial\beta}\ln Z_1$代入熵式得$S=k(N\ln N+N\alpha+\beta U)=k\left[N\ln N+\sum_l(\alpha+\beta\varepsilon_l)a_l\right]$,又$a_l=\omega_le^{-\alpha-\beta\varepsilon_l}\Rightarrow\alpha+\beta\varepsilon_l=\ln\frac{\omega_l}{a_l}$,得$S=k\left[N\ln N+\sum_la_l\ln\omega_l-\sum_la_l\ln a_l\right]$,与$\ln\Omega=N\ln N-\sum_la_l\ln a_l+\sum_la_l\ln\omega_l$比较得$S=k\ln\Omega$\\
\textbf{自由能}:$F=U-TS=-N\frac{\partial}{\partial\beta}\ln Z_1-TNk\left(\ln Z_1-\beta\frac{\partial}{\partial\beta}\ln Z_1\right)=-TNk\ln Z_1$\\
对\textbf{满足经典极限条件的玻色/费米系统}:\textbf{配分函数}:$Z_1=\sum_l\omega_le^{-\beta\varepsilon_l}$;\textbf{总粒子数}:$N=e^{-\alpha}Z_1$;\textbf{内能}:$U=-N\frac{\partial}{\partial\beta}\ln Z_1$;\textbf{广义力}:$Y=-\frac{N}{\beta}\frac{\partial}{\partial\beta}\ln Z_1$;(唯二和玻尔兹曼系统有差异的)\textbf{熵}:$S=Nk\left(\ln Z_1-\beta\frac{\partial}{\partial\beta}\ln Z_1\right)-k\ln N!=k\ln\frac{\Omega_{M.B.}}{N!}=k\ln\Omega_{B.E.}$;\textbf{自由能}:$F=-NkT\ln Z_1+kT\ln N!$\\
\textbf{经典系统配分函数}:$Z_1=\sum_le^{-\beta\varepsilon_l}\frac{\Delta\omega_l}{h_0^r}=\int e^{-\beta\varepsilon_l}\frac{d\omega_l}{h_0^r}\int\cdots\int e^{-\beta\varepsilon(p,q)}\frac{dq_1\cdots dq_rdp_1\cdots\cdots dp_r}{h_0^r}$\\
\textbf{总粒子数}:$N=e^{-\alpha}Z_1$;
\textbf{分布}:$a_l=e^{-\alpha-\beta\varepsilon_l}\frac{\Delta\omega_l}{h_0^r}=\frac{N}{Z_1}e^{-\beta\varepsilon_l}\frac{\Delta\omega_l}{h_0^r}$,$h_0^r$与$Z_1$中$h_0^{-r}$相消,故分布与$h_0^r$值无关\\
\textbf{内能}:$U=-N\frac{\partial}{\partial\beta}\ln Z_1$,与$h_0$取值无关;\quad
\textbf{广义力}:$Y=-\frac{N}{\beta}\frac{\partial}{\partial y}\ln Z_1$,与$h_0$取值无关\\
\textbf{熵}:$S=Nk\left(\ln Z_1-\beta\frac{\partial}{\partial\beta}\ln Z_1\right)=k\ln\Omega$,$h_0$取值不同将导致熵差一常数,这说明绝对熵的概念是量子理论的结果\\
\textbf{7.2 理想气体的物态方程}
一般气体满足经典极限条件,遵从玻尔兹曼分布\\
\textbf{单原子分子理想气体配分函数}:$Z_1=\sum_l\omega_le^{-\beta\varepsilon_l}=\sum_l\frac{dxdydzdp_xdp_ydp_z}{h^3}e^{-\beta\varepsilon_l}=\frac{1}{h^3}\idotsint\\e^{-\frac{\beta}{2m}(p_x^2+p_y^2+p_z^2)}dxdydzdp_xdp_ydp_z=\frac{V}{h^3}\left(\int_{-\infty}^{+\infty}e^{-\frac{\beta}{2m}p_x^2}dp_x\right)^3=V\left(\frac{2m\pi}{\beta h^2}\right)^{3/2}$\\
\textbf{压强/物态方程}:$p=\frac{N}{\beta}\frac{\partial}{\partial V}\ln Z_1=\frac{N}{V\beta}\Rightarrow pV=kTN=nkTN_A$,与$pV=nRT$比较得$R=kN_A$\\
\textbf{经典极限条件}:将单原子分子理想气体配分函数代入经典极限条件得$e^{\alpha}=\frac{Z}{N}=\frac{V}{N}\left(\frac{2m\pi kT}{h^2}\right)^{3/2}\gg1$,这说明$\frac{N}{V}$越小(气体越稀薄),$T$越高,分子质量越大,越趋于经典极限条件;经典极限条件还可表为$n\lambda^3\ll1$,其中分子德布罗意波的热波长$\lambda=\frac{h}{p}=\frac{h}{\sqrt{2m\varepsilon}}\approx\frac{h}{\sqrt{2\pi mkT}}$,分子数密度$n=\frac{N}{V}$,即分子热波长$\ll$分子间距或在体积$\lambda^3$内平均粒子数$\ll1$\\
\textbf{7.3 麦克斯韦速度分布律}:单位体积内$dv_xdv_ydv_z$范围内分子数$f(v_x,v_y,v_z)dv_xdv_ydv_z=n\left(\frac{m}{2\pi kT}\right)^{3/2}\\e^{-\frac{m}{2kT}(v_x^2+v_y^2+v_z^2)}dv_xdv_ydv_z$,其中\textbf{麦氏速度分布函数}$f(v_x,v_y,v_z)$满足$\iiint_{\infty}f(\vec{v})dv_xdv_ydv_z=n$;证明:$V$内$dp_xdp_ydp_z$范围内,分子平动状态数:$\frac{V}{h^3}dp_xdp_ydp_z$,分子数:$\frac{V}{h^3}e^{-\alpha-\frac{1}{2mkT}(p_x^2+p_y^2+p_z^2)}dp_xdp_ydp_z$,代入$e^{-\alpha}=\frac{N}{V}\left(\frac{h^2}{2\pi mkT}\right)^{3/2}$得$a=\frac{N}{(2\pi mkT)^{3/2}}e^{-\frac{1}{2kmT}(p_x^2+p_y^2+p_z^2)}dp_xdp_ydp_z=N\left(\frac{m}{2\pi kT}\right)^{3/2}e^{-\frac{m}{2kT}(v_x^2+v_y^2+v_z^2)}dv_xdv_ydv_z$,除以$V$即得\\
\textbf{速度空间球坐标中的麦氏速度分布律}:$f(v,\theta,\phi)v^2\sin\theta dvd\theta d\phi=n\left(\frac{m}{2\pi kT}\right)^{3/2}e^{-\frac{m}{2kT}v^2}v^2\sin\theta dvd\theta d\phi$;对立体角积分得\textbf{麦氏速率分布律}:$f(v)dv=\int_0^{2\pi}\int_{0}^{\pi}f(v,\theta,\phi)v^2\sin\theta dvd\theta d\phi=4\pi n\left(\frac{m}{2\pi kT}\right)^{3/2}e^{-\frac{m}{2kT}v^2}v^2dv$,其中\textbf{速率分布函数}$f(v)$满足$\int_0^{\infty}f(v)dv=n$\\
\textbf{最概然速率}:使速率分布函数取极大值的速率,$v_m=\sqrt{\frac{2kT}{m}}$;由$\frac{df(v)}{dv}=0\Rightarrow v\left(2-\frac{m}{kT}\right)e^{-\frac{m}{2kT}v^2}=0$且$\frac{d^2f(v)}{dv^2}<0\Rightarrow\left[\left(2-\frac{m}{kT}v^2\right)+v\left(-2\frac{m}{kT}v\right)+v\left(2-\frac{m}{kT}v^2\right)\left(-\frac{m}{kT}v\right)\right]e^{-\frac{m}{2kT}v^2}<0$得\\
\textbf{平均速率}:$\bar{v}=\frac{1}{n}\int vf(v)dv=4\pi\left(\frac{m}{2\pi kT}\right)^{3/2}\int_0^{\infty}e^{-\frac{m}{2kT}v^2}v^3dv=\sqrt{\frac{8kT}{\pi m}}$\\
\textbf{方均根速率}:$\overline{v^2}=\frac{1}{n}\int v^2f(v)dv=4\pi\left(\frac{m}{2\pi kT}\right)^{3/2}\int_0^{\infty}e^{-\frac{m}{2kT}v^2}v^4dv=\frac{3kT}{m}$;\textbf{平均平动动能}:$\bar{\varepsilon}=\frac{1}{2}m\overline{v^2}=\frac{3}{2}kT$\\
\textbf{碰壁数}:单位时间内碰到单位面积上的分子数$\Gamma=\frac{1}{4}n\bar{v}$;证明:$dt$内碰到$dA$上,$dv_xdv_ydv_z$范围内的$d\Gamma$个分子应位于以$dA$为底,$v(v_x,v_y,v_z)$为轴,$v_xdt$为高的柱体内,故$d\Gamma dAdt=fdv_xdv_ydv_zdA(v_x,t)\Rightarrow d\Gamma=v_xfdv_xdv_ydv_z\Rightarrow\Gamma=\int_{-\infty}^{+\infty}dv_y\int_{-\infty}^{+\infty}dv_z\int_0^{+\infty}fv_xdv_x=n\left(\frac{m}{2\pi kT}\right)^{3/2}\left[\int_{-\infty}^{+\infty}e^{-\frac{m}{2kT}v_y^2}dv_y\right]\int_0^{+\infty}e^{-\frac{m}{2kT}v_x^2}v_xdv_x=n\sqrt{\frac{kT}{2\pi m}}$\\
\textbf{压强}:$p=\frac{n_mRT}{V}$;证明:在$dt$内碰到$dA$上,$dv_xdv_ydv_z$范围内的分子受到器壁冲量$dI=2mv_xd\Gamma dAdt$使分子速度由$v_x$变为$-v_x$,压强$dp=(dI/dt)/dA=2mv_x\Gamma=2mv_x^2fdv_xdv_ydv_z\Rightarrow p=2m\int_{-\infty}^{+\infty}dv_y\int_{-\infty}^{+\infty}dv_z\int_0^{+\infty}fv_x^2dv_x$,由$\int_0^{+\infty}e^{-\alpha x^2}x^2dx=\frac{\sqrt{\pi}}{4}\alpha^{-3/2}$得$p=2mn\left(\frac{m}{2\pi kT}\right)^{3/2}\left[\int_{-\infty}^{+\infty}e^{-\frac{m}{2kT}v_y^2}dv_y\right]\int_0^{+\infty}e^{-\frac{m}{2kT}v_x^2}dv_x=nkT=\frac{N}{V}kT$\\
\textbf{7.4 能量均分定理}:对处在温度为$T$的平衡态的经典系统,粒子能量中的每一平方项的平均值为$kT/2$;证明:视系统为经典系统,粒子总能量为动能与势能之和$\varepsilon=\varepsilon_p+\varepsilon_q=\frac{1}{2}\sum_{i=1}^ra_ip_i^2+\frac{1}{2}\sum_{i=1}^{r'}b_iq_i^2+\varepsilon_q'(q_{r'+1},\cdots,q_r)$,其中$a_i,b_i>0$且与$p_1,\cdots,p_r,q_1,\cdots,q_r'$无关,故分布:$a=\frac{1}{h^r}e^{-\alpha-\beta\varepsilon}dp_1\cdots dp_rdq_1\cdots dq_r=\frac{N}{Z_1h^r}e^{-\beta\varepsilon}dp_1\cdots dp_rdq_1\cdots dq_r$,其中配分函数$Z_1=\frac{1}{h^r}\idotsint_{-\infty}^{+\infty}e^{-\beta\varepsilon}dp_1\cdots dp_rdq_1\cdots dq_r$,能量表达式中任意平方项平均值$\overline{\frac{1}{2}a_ip_i^2}=\frac{1}{N}\idotsint_{-\infty}^{+\infty}\frac{1}{2}a_ip_i^2e^{-\alpha-\beta\varepsilon}\frac{1}{h^r}dp_1\cdots dp_rdq_1\cdots dq_r=\frac{1}{Z_1h^r}\idotsint_{-\infty}^{+\infty}\frac{1}{2}a_ip_i^2e^{-\beta\varepsilon'-\frac{\beta}{2}a_ip_i^2}dp_1\cdots dp_rdq_1\cdots dq_r=\frac{1}{Z_1h^r}\idotsint_{-\infty}^{+\infty}dp_1\cdots dp_{i-1}\\dp_{i+1}\cdots dp_rdq_1\cdots dq_r\int_{-\infty}^{+\infty}\frac{1}{2}a_ip_i^2e^{-\frac{\beta}{2}a_ip_i^2}dp_i$,由$\int_{-\infty}^{+\infty}\frac{1}{2}a_ip_i^2e^{-\frac{\beta}{2}a_ip_i^2}dp_i=-\frac{1}{2\beta}\int_{-\infty}^{+\infty}\\p_id\left(e^{-\frac{\beta}{2}a_ip_i^2}\right)=\frac{1}{2\beta}\int_{-\infty}^{+\infty}e^{-\frac{\beta}{2}a_ip_i^2}dp_i$得$\overline{\frac{1}{2}a_ip_i^2}=\frac{1}{Z_1h^r}\idotsint_{-\infty}^{+\infty}e^{-\beta\varepsilon'-\frac{\beta}{2}a_ip_i^2}dp_1\cdots dp_r\\dq_1\cdots dq_r=\frac{1}{2\beta Z_1h^r}\idotsint_{-\infty}^{+\infty}e^{-\beta\varepsilon}dp_1\cdots dp_rdq_1\cdots dq_r=\frac{1}{2\beta}=\frac{1}{2}kT$,同理$\overline{\frac{1}{2}b_iq_i^2}=\frac{1}{2}kT$\\
对\textbf{单原子分子},\textbf{质心平动动能}:$\varepsilon=\frac{1}{2m}(p_x^2+p_y^2+p_z^2)$;\textbf{分子平均能量}:$\bar{\varepsilon}=\frac{3}{2}kT$;\textbf{总内能}:$U=\bar{\varepsilon}N=\frac{3}{2}NkT$;\textbf{定容热容}:$C_V=\frac{dU}{dT}=\frac{3}{2}Nk$;\textbf{定压热容}:$C_p=C_V+Nk=\frac{5}{2}Nk$;\textbf{热容比}:$\frac{C_p}{C_V}=\frac{5}{3}$;未考虑原子内电子,需量子理论再解释\\
对\textbf{双原子分子},\textbf{分子能量}为平动能量/绕质心转动能量/原子相对运动能量/相互作用能量之和$\varepsilon=\frac{1}{2m}(p_x^2+p_y^2+p_z^2)+\frac{1}{I}(p_{\theta}^2+\frac{1}{\sin^2\theta}p_{\varphi}^2)+\frac{1}{2m_{\mu}}p_r^2+u(r)$;不考虑后两项,\textbf{分子平均能量}:$\bar{\varepsilon}=\frac{5}{2}kT$;\textbf{总内能}:$U=\bar{\varepsilon}N=\frac{5}{2}NkT$;\textbf{定容热容}:$C_V=\frac{dU}{dT}=\frac{5}{2}Nk$;\textbf{定压热容}:$C_p=\frac{7}{2}Nk$;\textbf{热容比}:$\gamma=\frac{7}{5}$;无法解释低温下H$_2$的性质,未考虑两原子的相对运动\\
对\textbf{固体中的原子振动},\textbf{原子能量}:$\varepsilon=\frac{1}{2m}(p_x^2+p_y^2+p_z^2)+\frac{1}{2}m\omega_x^2q_x^2+\frac{1}{2}\omega_y^2q_y^2+\frac{1}{2}m\omega_z^2q_z^2$;\textbf{原子平均能量}:$\bar{\varepsilon}=3kT$;\textbf{总内能}$U=\bar{\varepsilon}N=3NkT$;\textbf{热容}:$C_V=\frac{dU}{dT}=3Nk$;低温下理论与实验不符合,实验发现$C_V\downarrow$随$T\downarrow$,终$\rightarrow0$,$3$K以上自由电子的热容可忽略\\
对\textbf{平衡辐射},辐射场可分解为一系列满足周期性边界条件的单色平面波的叠加,$E=E_0e^{i(\bm{k}\cdot\bm{r}-\omega t)}$,其中$\omega=ck$,$k_{x/y/z}=\frac{2\pi}{L}n_x,n_{x/y/z}=0,\pm1,\cdots$,$E_0$有$2$个相垂且垂直于$\bm{k}$的偏振方向;具有一定波矢$k$和一定偏振的单色平面波可视为一自由度,在$V$内$dk_xdk_ydk_z$范围内,振动自由度:$\frac{dk_x}{2\pi/L}\frac{dk_y}{2\pi/L}\frac{dk_z}{2\pi/L}=Vdk_xdk_ydk_z/4\pi^3$;$V$内$\omega\sim\omega+d\omega$范围内,振动自由度:$D(\omega)d\omega=\frac{V}{4\pi^3}\left(\frac{\omega}{2\pi c}\right)^2d\left(\frac{\omega}{2\pi c}\right)\int_0^{\pi}\sin\theta d\theta\\\int_0^{2\pi}d\phi=\frac{V}{\pi^2c^3}\omega^2d\omega$;该范围内总振动能量:$U_{\omega}d\omega=D(\omega)kTd\omega=\frac{V}{\pi^2c^3}\omega^2kTd\omega$(瑞利·金斯公式);低频段与实验符合,高频段偏差,积分得总能量发散,与斯特藩-玻尔兹曼定律不符,因经典电动力学辐射场有无穷多个振动自由度\\
\textbf{7.5 理想气体的内能和热容}
忽略电子运动,\textbf{双原子分子能量}:$\varepsilon=\varepsilon^t+\varepsilon^v+\varepsilon^r$,其中$^t$,$^v$,$^r$--平动/振动/转动分量;\textbf{配分函数}$Z_1=\sum_l\omega_le^{-\beta\varepsilon_l}=\sum_{t,v,r}\omega^t\omega^v\omega^re^{-\beta(\varepsilon^t+\varepsilon^v+\varepsilon^r)}=\sum_t\omega^te^{-\beta\varepsilon^t}\sum_v\omega^ve^{-\beta\varepsilon^v}\sum_r\omega^re^{-\beta\varepsilon_r}=Z_1^tZ_1^vZ_1^r$;\textbf{内能}:$U=-N\frac{\partial}{\partial\beta}\ln Z_1=-N\frac{\partial}{\partial\beta}(\ln Z_1^t+\ln Z_1^v+\ln Z_1^r)=U^t$;\textbf{定容热容}:$C_V=\frac{\partial U}{\partial T}=C_V^t+C_V^v+C_V^r$\\
\textbf{平动配分函数}:$Z_V^r=V\left(\frac{2m\pi}{\beta h^2}\right)^{3/2}$;\textbf{内能}:$U^t=-N\frac{\partial}{\partial\beta}\ln Z_1^t=\frac{3N}{2\beta}=\frac{3}{2}NkT$;\textbf{定容热容}:$C_V^t=\frac{3}{2}Nk$\\
视相对振动的两原子为线性谐振子,\textbf{振动能级}:$\varepsilon_n^v=\left(n+\frac{1}{2}\right)\hbar\omega$;\textbf{配分函数}:$Z_1^v=\sum_{n=0}^{\infty}e^{-\beta\hbar\omega(n+\frac{1}{2})}=\frac{e^{-\beta\hbar\omega/2}}{1-e^{-\beta\hbar\omega}}$;\textbf{内能}:$U^V=-N\frac{\partial}{\partial\beta}\ln Z_1^v=\frac{N\hbar\omega}{2}+\frac{N\hbar\omega}{e^{\beta\hbar\omega}-1}=\frac{Nk\theta_v}{2}+\frac{Nk\theta_v}{e^{\theta_v/T}-1}$,其中第一项为$N$个振子的零点能,与温度无关,第二项为$T$下$N$个振子的热激发能,\textbf{振动特征温度}:$\theta_v=\frac{\hbar\omega}{k}$;\textbf{振动贡献定容热容}:$C_V=\left(\frac{\partial U}{\partial T}\right)_V=Nk\left(\frac{\hbar\omega}{kT}\right)^2\frac{e^{\hbar\omega/kT}}{(e^{\hbar\omega/kT}-1)^2}=Nk\left(\frac{\theta}{T}\right)^2\frac{e^{\theta_v/T}}{(e^{\theta_v/T}-1)^2}$;常温下$T\ll\theta_v$,$U^v=\frac{Nk\theta_v}{2}+Nk\theta_ve^{-\theta_v/T}$,$C_V^v=Nk\left(\frac{\theta_v}{T}\right)^2e^{-\theta_v/T}$,这说明常温下$C_V^v\rightarrow0$,不参与能量均分\\
对\textbf{异核双原子分子}(无全同性影响):\textbf{转动能量}:$\varepsilon^r=\frac{l(l+1)}{2I}\hbar^2,l=0,1,\cdots$;\textbf{简并度}:$\omega_l=2l+1$;\textbf{转动配分函数}:$Z_1^r=\sum_{l=0}^{\omega}(2l+1)e^{-l(l+1)\hbar^2/2IkT}=\sum_1^r(2l+1)e^{-l(l+1)\theta_r/T}$,其中\textbf{转动特征温度}:$\theta_r=\frac{\hbar^2}{2Ik}$;常温下$T\gg\theta_r$,令$x=l(l+1)\frac{\theta_r}{T}$,$Z_1^r=\int_0^{+\infty}(2l+1)e^{-x}\frac{dx}{(2l+1)(\theta_r/T)}=\frac{T}{\theta_r}=\frac{2I}{\beta\hbar^2}$,\textbf{转动贡献内能}:$U^r=-N\frac{\partial}{\partial\beta}\ln Z_1^r=NkT$,\textbf{转动贡献定容热容}:$C_V^r=Nk$,与经典统计能量均分定理结论一致\\
对\textbf{同核双原子分子}:以氢为例,常温下含$3/4$两原子自旋平行,$l$为奇数的正氢,$1/4$自旋反平行,$l$偶的仲氢,\textbf{正氢配分函数}:$Z_{1o}^r=\sum_{l=1,3,\cdots}(2l+1)e^{-l(l+1)\theta_r/T}$,\textbf{仲氢配分函数}:$Z_{1p}^r=\sum_{l=0,2,\cdots}(2l+1)e^{-l(l+1)\theta_r/T}$;\textbf{转动贡献内能}:$U^r=U_o^r+U_p^r=-\frac{3}{4}N\frac{\partial}{\partial\beta}\ln Z_{1o}^r-\frac{1}{4}N\frac{\partial}{\partial\beta}\ln Z_{1p}^r$;常温下$T\gg\theta_r$,$Z_{1o}^r\approx Z_{1p}^r\approx\frac{1}{2}\sum_{l=0,1,\cdots}\approx\frac{1}{2}\int_0^{\infty}(2l+1)\frac{e^{-x}dx}{(2l+1)(\theta_r/T)}=\frac{1}{2}\frac{T}{\theta_r}=\frac{I}{\beta\hbar^2}$,\textbf{转动贡献内能}:$U^r\approx-N\frac{\partial}{\partial\beta}\ln Z_{1o}^r=-N\frac{\partial}{\partial\beta}\ln\frac{I}{\beta\hbar^2}=\frac{N}{\beta}=NkT$,\textbf{转动贡献热容}:$C_V^r=Nk$,与能量均分定理的结果一致,低温下不适用,需严格计算级数求和\\
\textbf{7.6 理想气体的熵}
对\textbf{单原子理想气体},\textbf{经典统计理论得熵}:$S=\frac{3}{2}Nk\ln T+Nk\ln V+\frac{3}{2}Nk\left[1+\ln\left(\frac{2\pi mk}{h_0^2}\right)\right]$,非绝对熵,随$\hbar_0$不同而不同,且不满足广延量要求\\
经典极限条件下,\textbf{量子理论得熵}:$S=k\ln\frac{\Omega_{\text{M.B.}}}{N!}=Nk(\ln Z_1-\beta\frac{\partial}{\partial\beta}\ln Z_1)-k\ln N!$,代入$Z_1$用斯特林公式得$S=\frac{3}{2}Nk\ln T+Nk\ln\frac{V}{N}+\frac{3}{2}Nk\left[\frac{5}{3}+\ln\left(\frac{2\pi mk}{h^2}\right)\right]$,是绝对熵,满足广延量要求\\
对与凝聚相达平衡的饱和蒸汽,视为理想气体,将理想气体物态方程代入熵式得$\ln p=\frac{5}{2}\ln T+\frac{5}{2}+\ln\left[k^{5/2}\left(\frac{2\pi m}{h^2}\right)^{3/2}\right]-\frac{S_{\text{gas}}}{Nk}$,用克拉珀龙方程$S_{\text{vap}}-S_{\text{con}}=\frac{L}{T}$,足够低$T$下,$S_{\text{con}}\ll\frac{L}{T}$,$\ln p=-\frac{L}{RT}+\frac{5}{2}\ln T+\frac{5}{2}+\ln\left[k^{5/2}\left(\frac{2\pi m}{h^2}\right)^{3/2}\right]$(\textbf{萨库尔-铁特罗特公式}),与实验相符\\
\textbf{理想气体的化学势}:一个分子的化学势:$\mu=\left(\frac{\partial F}{\partial N}\right)_{T,N}$,代入$F$式得$\mu=-kT\ln\frac{Z_1}{N}$,代入$Z_1^t$得$\mu=kT\ln\left[\frac{N}{V}\left(\frac{h^2}{2\pi mkT}\right)^{3/2}\right]$,经典极限条件下$\frac{V}{N}\left(\frac{2m\pi kT}{h^2}\right)^{3/2}\gg1$,故$\mu<0$\\
\textbf{7.7 固体热容的爱因斯坦理论}
视固体中原子为$3N$个相同振动频率的振子,其能级:$\varepsilon_n=\hbar\omega(n+\frac{1}{2})$,$n=0,1,\cdots$,振子定域,故遵从玻尔兹曼分布,\textbf{配分函数}:$Z_1=\sum_{n=0}^{\infty}\omega_le^{-\beta\hbar\omega(n+\frac{1}{2})}=\frac{e^{-\frac{1}{2}\beta\hbar\omega}}{1-e^{-\beta\hbar\omega}}$,\textbf{内能}:$U=-3N\frac{\partial}{\partial\beta}\ln Z_1=-3N\frac{\partial}{\partial\beta}\ln Z_1=3N\frac{\hbar\omega}{2}+\frac{3N\hbar\omega}{e^{\beta\hbar\omega}-1}$,其中第一项为零点能,第二项为热激发能;定义\textbf{特征温度$\theta_{E}$}满足$k\theta_E=\hbar\omega$,\textbf{内能}:$U=\frac{3}{2}Nk\theta+\frac{3Nk\theta_E}{e^{\theta_E/T}-1}$,\textbf{热容}:$C_V=\left(\frac{\partial U}{\partial T}\right)_V=3Nk\left(\frac{\theta_E}{T}\right)^2\frac{e^{\theta_E/T}}{(e^{\theta_E/T}-1)^2}$;当$T\gg\theta_E$,$e^{\theta_E/T}-1\approx\frac{\theta_E}{T}$,$C_V=3Nk$;当$T\ll\theta_E$,$e^{\theta_E/T}-1\approx e^{\theta_E/T}$,$C_V\approx3Nk\left(\frac{\theta}{T}\right)^2e^{-\theta_E/T}\rightarrow0$;因相同振动频率的假设过度简化,仅定性符合实验,定量上有差异\\
\textbf{7.8 顺磁固体的热容量}
设磁性离子定域在晶体特定各点上,密度低而可忽略相互作用,可视为定域近独立粒子系统,遵从玻尔兹曼分布\\
设离子总角动量量子数$1/2$,磁矩$\mu=-\frac{e\hbar}{2m}$在外场$B$中能量可能取值:$-\mu B$,$\mu B$,\textbf{配分函数}:$Z_1=e^{\beta\mu B}+e^{-\beta\mu B}$,\textbf{磁化强度}:$M=\frac{n}{\beta}\frac{\partial}{\partial B}\ln Z_1=n\mu\frac{e^{\beta\mu B}-e^{-\beta\mu B}}{e^{\beta\mu B}+e^{-\beta\mu B}}=n\mu\tanh\left(\frac{\mu B}{kT}\right)$;弱场或高温下,$\frac{\mu B}{kT}\ll1$,$\tanh\frac{\mu B}{kT}\approx\frac{\mu B}{kT}$,$M=\frac{n\mu^2}{kT}B=\chi H$,其中磁化率$\chi=n\mu^2\mu_0/kT$;强场或低温下,$\frac{\mu B}{kT}\gg1$,$e^{\mu B/kT}\gg e^{-\mu B/kT}$,$M=n\mu$,所有自旋磁矩均沿外场方向,磁化达饱和\\
\textbf{单位体积的内能}:$u=-n\frac{\partial}{\partial\beta}\ln Z_1=-n\mu B\tanh\frac{\mu B}{kT}=-MB$亦顺磁体在外场中的势能\\
\textbf{单位体积的熵}:$s=nk\left(\ln Z_1-\beta\frac{\partial}{\partial\beta}\ln Z_1\right)=nk\left[\ln2+\ln\cosh\left(\frac{\mu B}{kT}\right)-\left(\frac{\mu B}{kT}\right)\tanh\left(\frac{\mu B}{kT}\right)\right]$;弱场/高温下,$\frac{\mu B}{kT}\ll1$,$\tanh\left(\frac{\mu B}{kT}\right)\approx\frac{\mu B}{kT}$,$\ln\cosh\left(\frac{\mu B}{kT}\right)\approx\ln\left[1+\frac{1}{2}\left(\frac{\mu B}{kT}\right)^2\right]\approx\frac{1}{2}\left(\frac{\mu B}{kT}\right)^2$,$s=nk\ln2=k\ln2^n$,磁矩沿/逆外场方向的概率近似相等,每个磁矩各有$2$个可能的状态,单位体积微观态数:$2^n$;强场/低温下,$\frac{\mu B}{kT}\gg1$,$\cosh\left(\frac{\mu B}{kT}\right)\approx\frac{1}{2}e^{\mu B/kT}$,$\tanh\left(\frac{\mu B}{kT}\right)\approx1$,$s=0$,磁矩均沿外场方向,系统仅$1$微观态\\
\textbf{7.9 负温度状态}
由$dU=TdS+Ydy$有$\frac{1}{T}=\left(\frac{\partial S}{\partial U}\right)_y$;当系统的内能增加而熵反而减小时,系统就处在\textbf{负温度状态}\\
对\textbf{孤立的核自旋系统},设核自旋量子数$\frac{1}{2}$,外磁场$B$作用下,能量有$2$个可能值:$\pm\varepsilon=\pm\mu B=\pm\frac{e\hbar}{2mB}$,\textbf{总磁矩数}:$N=N_++N_-$,\textbf{总能量}:$E=(N_+-N_-)\varepsilon$,其中$N_{\pm}=\frac{N}{2}(1\pm\frac{E}{N\varepsilon})$--能量为$\varepsilon_{\pm}$的磁矩数,\textbf{熵}:$S=k\ln\Omega=k\ln\frac{N!}{N_+!N_-!}$,用斯特林公式得$S=k(N\ln N-N_+\ln N_+-N_-\ln N_-)=Nk\left[\ln2-\frac{1}{2}\left(1+\frac{E}{N\varepsilon}\right)\ln\left(1+\frac{E}{N\varepsilon}\right)-\frac{1}{2}\left(1-\frac{E}{N\varepsilon}\right)\ln\left(1-\frac{E}{N\varepsilon}\right)\right]$,从而$\frac{1}{T}=\left(\frac{\partial S}{\partial E}\right)_B=\frac{k}{2\varepsilon}\ln\frac{N\varepsilon-E}{N\varepsilon+E}$;当$E<0$,$T>0$,正温状态;当$E<0$,$T<0$,负温状态\\
\textbf{负温度的性质}:处于负温状态的系统能量高于正温状态;当$1$负温状态系统与$1$正温状态系统热接触,热量从负温处传至正温处;两个结构相同,分别处于$\pm273$K的系统热接触达平衡后共同温度为$\pm\infty$K($\pm\infty$K等温)\\
\textbf{负温度的条件}:1.粒子能级必有上限,否则系统可能微观态数$\uparrow$随$E\uparrow$,即$S$为关于$E$单增,$T>0$;2.负温系统需与正温系统隔绝,或系统本身达平衡的弛豫时间$\ll$系统与正温系统达平衡的弛豫时间
\end{multicols}

\begin{multicols}{2}
\noindent\textbf{Chap8 玻色统计和费米统计}
\textbf{8.1 热力学量的统计表达式}
\textbf{非简并性条件}:$e^{\alpha}=\frac{V}{N}\left(\frac{2\pi nkT}{h^2}\right)\ll1$;\textbf{非简并性气体}:满足非简并性条件的气体,可用玻尔兹曼分布;\textbf{简并性气体}:不满足非简并性条件的气体,需用玻色/费米分布\\
\textbf{玻色分布}:$a_l=\frac{\omega_l}{e^{\alpha+\beta\varepsilon_l}-1}$;
\textbf{系统平均总粒子数}:$\bar{N}=\sum_l\frac{\omega_l}{e^{\alpha+\beta\varepsilon_l}-1}$;
\textbf{系统平均内能}:$U=\sum_l\frac{\varepsilon_l\omega_l}{e^{\alpha+\beta\varepsilon_l}-1}$\\
\textbf{巨配分函数}:$\Xi=\prod_l\Xi_l=\prod_l[1-e^{-\alpha-\beta\varepsilon_l}]^{-\omega_l}$,其对数:$\ln\Xi=-\sum_l\omega_l\alpha\ln(1-e^{-\alpha-\beta\varepsilon_l})$\\
\textbf{系统平均总粒子数}:$\overline{N}=-\frac{\partial}{\partial\alpha}\ln\Xi$;
\textbf{内能}:$U=\sum_l\varepsilon_la_l=-\frac{\partial}{\partial\beta}\ln\Xi$\\
\textbf{广义力}:$Y=\sum_l\frac{\partial\varepsilon_l}{\partial y}=\sum_l\frac{\partial\varepsilon_l}{\partial y}\frac{\omega_l}{e^{\alpha+\beta\varepsilon_l}-1}$;\textbf{压强}:$p=\frac{1}{\beta}\frac{\partial}{\partial V}\ln\Xi$\\
$\beta\left(dU-Ydy+\frac{\alpha}{\beta}d\overline{N}\right)=-\beta d\left(\frac{\partial\ln\Xi}{\partial\beta}\right)+\frac{\partial\ln\Xi}{\partial y}dy-a\alpha d\left(\frac{\partial\ln\Xi}{\partial\alpha}\right)$,因$\ln\Xi$为$\alpha$,$\beta$,$y$的函数,其全微分:$d(\ln\Xi)=\frac{\partial\Xi}{\partial\alpha}d\alpha+\frac{\partial\ln\Xi}{\partial\beta}d\beta+\frac{\partial\ln\Xi}{\partial y}dy$,代入前式得$\beta\left(dU-Ydy+\frac{\alpha}{\beta}\frac{\partial}{\partial\beta}\overline{N}\right)=d\left(\ln\Xi-\alpha\frac{\partial}{\partial\alpha}\ln\Xi-\beta\frac{\partial}{\partial\beta}\ln\Xi\right)$,比较开系热力学基本方程$\frac{1}{T}(dU-Ydy-\mu d\overline{N})=dS$得$\beta=\frac{1}{kT}$,$\alpha=-\frac{\mu}{kT}$,\textbf{熵}:$S=k(\ln\Xi-\alpha\frac{\partial}{\partial\alpha}\ln\Xi-\beta\frac{\partial}{\partial}\beta\ln\Xi)=k(\ln\Xi+\alpha\overline{N}+\beta U)$;代入$\ln\Xi$,$\ln\Omega_{\text{B.E.}}$得:$S=k\ln\Omega$\\
\textbf{费米分布}:$a_l=\frac{\omega_l}{e^{\alpha+\beta\varepsilon_l}+1}$;
\textbf{巨配分函数}:$\Xi=\sum_l[1+e^{-\alpha-\beta\varepsilon_l}]^{\omega_l}$,其对数:$\ln\Xi=\sum_l\omega_l\ln(1+e^{-\alpha-\beta\varepsilon_l})$\\
费米分布各热力学量用$\Xi$表示的表达式与玻色分布相同\\
$\ln\Xi$--$T$,$V$,$\mu$的特性函数;\textbf{巨热力学式$J$与$\Xi$的关系}:$J=-kT\ln\Xi$\\
\textbf{8.2 弱简并理想玻色气体和费米气体}
$\alpha^{-\alpha}$或$n\lambda^3$虽小但不可忽略,用费米(对应上面的符号)/玻色分布
忽略分子内部结构,仅考虑平动自由度,分子能量:$\varepsilon=\frac{1}{2m}(p_x^2+p_y^2+p_z^2)$,$V$内$\varepsilon\sim\varepsilon+d\varepsilon$范围内分子可能的微观态数:$D(\varepsilon)d\varepsilon=g\frac{2\pi V}{h^3}(2m)^{3/2}\varepsilon^{1/2}d\varepsilon$,其中$g$--自旋导致的简并度,拉氏乘子$\alpha$满足\textbf{粒子数}:$N=g\frac{2\pi V}{h^3}(2m)^{3/2}\int_0^{\infty}\frac{\varepsilon^{1/2}}{e^{\alpha+\beta\varepsilon}\pm1}$,\textbf{内能}:$U=g\frac{2\pi V}{h^3}(2m)^{3/2}\int_0^{\infty}\frac{\varepsilon^{3/2}d\varepsilon}{e^{\alpha+\beta\varepsilon}\pm1}$,设$x=\beta\varepsilon$,$N=g\frac{2\pi V}{h^3}(2mkT)^{3/2}\int_0^{\infty}\frac{x^{1/2}}{e^{\alpha+x}\pm1}$,$U=g\frac{2\pi V}{h^3}(2mkT)^{3/2}kT\int_0^{\infty}\frac{x^{3/2}dx}{e^{\alpha+x}\pm1}$,被积函数分母$\frac{1}{e^{\alpha+x}\pm1}=\frac{1}{e^{\alpha+x}(1\pm e^{-\alpha-x})}$,当$e^{-\alpha}$小,$\frac{1}{e^{\alpha+x}\pm1}\approx e^{-\alpha-x}(1\mp e^{-\alpha-x})$,回代得$N=g\left(\frac{2\pi mkT}{h^2}\right)^{3/2}Ve^{-\alpha}(1\mp 2^{-3/2}e^{-\alpha})$,$U=\frac{3}{2}g\left(\frac{2\pi mkT}{h^2}\right)^{3/2}VkTe^{-\alpha}(1\mp2^{-5/2}e^{-\alpha})=\frac{3}{2}NkT(1\pm2^{-5/2}e^{-\alpha})$,$e^{-\alpha}$零级近似$=\frac{1}{g}\frac{N}{V}\left(\frac{h^2}{2\pi mkT}\right)^{3/2}$,$U=\frac{3}{2}NkT\left[1\pm2^{-5/2}\frac{1}{g}\frac{N}{V}\left(\frac{h^2}{2\pi mkT}\right)^{3/2}\right]=\frac{3}{2}NkT\left(1\pm2^{-5/2}n\lambda^3\right)$,其中第一项为内能,第二项为由微观粒子全同性原理引起的量子统计关联导致的附加内能,费米气体的附加内能$>0$,玻色气体的附加内能$<0$,这说明量子统计关联使费米粒子间出现等效的排斥作用,玻色粒子间出现等效的吸引作用\\
\textbf{8.3 玻色-爱因斯坦凝聚}:极低温下的玻色气体中,宏观量级的粒子在最低能级凝聚\\
\textbf{玻色分布}:$a_l=\frac{\omega_l}{e^{(\varepsilon_l-\mu)/kT}-1}$,取最低能级为能量零点,$\varepsilon_0=0$,因任一能级上粒子数$>0$,$\mu<0$,$\mu$由$\frac{1}{V}\sum_l\frac{\omega_l}{e^{(\varepsilon_l-\mu)/kT}-1}=\frac{N}{V}=n$确定为$T$和粒子数密度$n$的函数,足够高$T$下由$D(\varepsilon)d\varepsilon=\frac{2\pi V}{h^3}(2m)^{3/2}\varepsilon^{1/2}d\varepsilon$求和用积分代替,$\frac{2\pi}{h^3}(2m)^{3/2}\int_0^{\infty}\frac{\varepsilon^{1/2}d\varepsilon}{e^{(\varepsilon-\mu)/kT}-1}=n$,$\mu\uparrow$随$T\downarrow$,当$T\rightarrow T_c$,$\mu\rightarrow-0$,$e^{-\frac{\mu}{kT_c}}\rightarrow1$,\textbf{临界温度$T_c$}满足$\frac{2\pi}{\hbar^3}(2m)^{3/2}\int_0^{\infty}\frac{\varepsilon^{1/2}d\varepsilon}{e^{\varepsilon/kT}-1}=n$,设$x=\frac{\varepsilon}{kT}$,上式化为$\frac{2\pi}{h^3}(2mkT_c)^{3/2}\int_0^{\infty}\frac{x^{1/2}dx}{e^x-1}=n\Rightarrow T_c=\frac{2\pi}{2.612^{2/3}}\frac{\hbar^2}{mk}n^{2/3}$,当$T<T_c$,$\mu=0$,$\frac{2\pi}{h^3}(2m)^{3/2}\int_0^{\infty}\frac{\varepsilon^{1/2}d\varepsilon}{e^{(\varepsilon-\mu)/kT}-1}<n$,产生这一矛盾是因为用积分代替求和,忽略了$\varepsilon=0$项,足够高$T$下,能级$\varepsilon=0$上粒子数相比总粒子数为一可忽略的小量,足够低$T$下,能级$\varepsilon=0$上粒子数可观,不可忽略,当$T<T_c$,$n_0(T)+\frac{2\pi}{h^3}(2m)^{3/2}\int_0^{\infty}\frac{\varepsilon^{1/2}d\varepsilon}{e^{\varepsilon/kT}-1}=n$,其中$n_0(T)=n\left[1-\left(\frac{T}{T_c}\right)^{3/2}\right]$--能级$\varepsilon=0$上粒子数密度,第二项--各激发能级上粒子数密度$n_{\varepsilon>0}=\frac{2\pi}{h^3}(2mkT)^{3/2}\int_0^{\infty}\frac{x^{1/2}dx}{e^x-1}=n\left(\frac{T}{T_c}\right)^{3/2}$;\textbf{玻色-爱因斯坦凝聚}:当$T<T_c$与总粒子数密度相同数量级粒子数密度的玻色粒子在能级$\varepsilon=0$上聚集;此时粒子动量$=0$,故又称在动量空间中的凝结\\
\textbf{当$T<T_c$,理想玻色气体内能}:$U=\frac{2\pi V}{h^3}(2m)^{3/2}\int_0^{\infty}\frac{\varepsilon^{3/2}d\varepsilon}{e^{\varepsilon/kT}-1}=\frac{2\pi V}{h^3}(2m)^{3/2}(kT)^{5/2}\int_0^{\infty}\frac{x^{3/2}dx}{e^x-1}=0.770NkT\left(\frac{T}{T_c}\right)^{3/2}$,\textbf{定容热容}:$\left(\frac{\partial U}{\partial T}\right)_V=1.925Nk\left(\frac{T}{T_c}\right)^{3/2}\propto T^{3/2}$,当$T=T_c$,$C_{V\max}=1.925Nk$,当$T\uparrow>T_c$,$C_V\rightarrow\frac{3}{2}Nk$\\
\textbf{弱作用玻色气体凝聚}:临界条件:$n\lambda^3=n\left(\frac{h}{\sqrt{2\pi mkT_c}}\right)=2.612$;当热波长与分子间平均距离具有相同量级,量子关联起决定性作用,凝聚出现的条件:$n\lambda^3\geq2.612$\\
%磁光陷阱
\textbf{8.4 光子气体}
将空窑内辐射场分解为一系列单色平面波的叠加,德布罗意关系:$\varepsilon=\hbar\omega$,$\varepsilon=\hbar\omega$,色散关系:$\omega=ck$,能量动量关系:$\varepsilon=cp$;光子--玻色子遵从玻色分布,因窑壁不断发射和吸收光子,光子数不守恒,故无$N$的约束而仅有$E$的约束,仅引入拉氏乘子$\beta$,$\alpha=0$,这说明平衡态下光子气体的化学势$=0$,光子自旋量子数$=1$,自旋在动量方向有$2$个可能值,相当于左/右圆偏振,$V$内$p\sim p+dp$范围内,光子量子态数:$\frac{8\pi V}{h^3}p^2dp$,$\omega\sim\omega+d\omega$范围内,光子量子态数$\frac{V}{\pi^2c^3}\omega^2d\omega$,平均光子数:$\frac{V}{\pi^2c^3}\frac{\omega^2d\omega}{e^{\hbar\omega/kT}-1}$,\textbf{普朗克公式}:$V$内$d\omega$内,辐射场的内能$U(\omega,T)d\omega=\frac{V}{\pi^2c^3}\frac{\hbar\omega^3d\omega}{e^{\hbar\omega/kT}-1}$;低频范围,$\frac{\hbar\omega}{kT}\ll1$,$e^{\hbar\omega/kT}\approx1+\frac{\hbar\omega}{kT}$,普氏公式近似为\textbf{瑞利-金斯公式}:$U(\omega,T)d\omega=\frac{V}{\pi^2c^3}\omega^3kTd\omega$;高频范围,$\frac{\hbar\omega}{kT}\gg1$,$e^{\hbar\omega/kT}\gg1$,普氏公式近似为\textbf{维恩公式}:$U(\omega,T)=\frac{V}{\pi^2c^3}\hbar\omega^3e^{-\hbar\omega/kT}d\omega$,\textbf{平衡辐射场总内能}:$U=\frac{V}{\pi^2c^3}\int_0^{\infty}\frac{\hbar\omega^3d\omega}{e^{\hbar\omega/kT}-1}d\omega$,设$y=\frac{\hbar\omega}{kT}$,$U=\frac{V\hbar}{\pi^2c^3}\left(\frac{kT}{\hbar}\right)\int_0^{\infty}\frac{y^3dy}{e^y-1}=\frac{\pi^2k^4}{15c^3\hbar^3}VT^4$(\textbf{斯-玻定律});能量密度极值满足$\frac{d}{dy}\left(\frac{y^3}{e^y-1}\right)=0\Rightarrow3-3e^{-y}=y\Rightarrow\frac{\hbar\omega_m}{kT}\approx2.822$\\
\textbf{光子气体的巨配分函数}:$\ln\Xi=-\sum_s\omega_l\ln(1-e^{-\beta\varepsilon_l})=-\frac{V}{\pi^2c^3}\int_0^{\infty}\omega^2\ln(1-e^{-\beta\hbar\omega})d\omega$,设$x=\frac{\hbar\omega}{kT}$,$\ln\Xi=-\frac{V}{\pi^2c^3}\left(\frac{1}{\beta\hbar}\right)^3\int_0^{\infty}x^2\ln(1-e^{-x})dx=\frac{\pi^2V}{45c^3}\left(\frac{1}{\beta\hbar}\right)^3$,\textbf{内能}:$U=-\frac{\partial}{\partial\beta}\ln\Xi=\frac{\pi^2k^4V}{15c^3\hbar^3}T^4$,\textbf{压强}:$p=-\frac{1}{\beta}\frac{\partial}{\partial V}\ln\Xi=\frac{\pi^2k^4}{45c^3\hbar^3}T^4=\frac{1}{3}\frac{U}{V}$,\textbf{熵}:$S=k(\ln\Xi-\beta U)=\frac{4}{45}\frac{\pi^2k^4V}{c^3\hbar^3}T^3V$,$S\rightarrow0$随$T\rightarrow0$,满足热$3$律,\textbf{辐射通量密度}:$J_u=\frac{cU}{4V}=\frac{\pi^2k^4}{60c^2\hbar^3}T^4$\\
\textbf{8.5 金属中的自由电子气体}
价电子脱离原子在金属中运动,称\textbf{共有电子},粗糙近似下,可视为封闭在金属内的自由电子,若将能量均分定理用于自由电子,一个自由电子对金属的热容有$3/2k$的贡献,不符实际,需用费米分布解释\\
对\textbf{费米分布},能量$\varepsilon$的$1$个量子态上平均电子数:$f=\frac{1}{e^{(\varepsilon-\mu)/kT}+1}$;$T=0$下,$f=1$,$\varepsilon<\mu_0$,$f=0$,$\varepsilon>\mu_0$;电子自旋在动量方向上的投影有两个值,$V$内$d\varepsilon$内电子的量子态数:$\frac{4\pi V}{h^3}(2m)^{3/2}\varepsilon^{1/2}d\varepsilon$,平均电子数:$\frac{4\pi V}{h^3}(2m)^{3/2}\frac{\varepsilon^{1/2}}{e^{(\varepsilon-\mu)/kT}+1}$,给定$N$,$T$,$V$下$\mu$满足\textbf{总电子数}:$N=\frac{4\pi V}{\hbar^3}(2m)^{3/2}\int_0^{\infty}\frac{\varepsilon^{1/2}}{e^{(\varepsilon-\mu)/kT}+1}$;$T=0$下,$N=\int_0^{\mu(0)}\Rightarrow$\textbf{费米能级}:$T=O$下,电子最大能量$\mu(0)=\frac{\hbar^2}{2m}\left(3\pi^2\frac{N}{V}\right)^{2/3}$,对常温$T$,$\mu(T)\sim\mu_0\ll kT$,$e^{\alpha}=e^{-\mu/kT}\ll1$,这说明金属中自由电子高度简并;\textbf{费米动量}:$p_F=\sqrt{2m\mu(0)}=(3\pi^2n)^{1/3}\hbar$;\textbf{费米速率}:$v_F=\frac{p_F}{m}$;\textbf{费米温度$T_F$}满足$kT_F=\mu(0)$\\
\textbf{总内能}:$U=\frac{4\pi V}{h^3}(2m)^{3/2}\int_0^{\mu(0)}\varepsilon^{3/2}d\varepsilon=\frac{3}{5}N\mu(0)$,\textbf{分子平均能量}:$\frac{3}{5}\mu(0)$,\textbf{电子气体的简并压}:$p(0)=\frac{2}{3}\frac{U(0)}{V}=\frac{2}{5}n\mu(0)$,极大,来源于泡利不相容原理和电子气体的高密度,被静电引力抵消\\
$T>0$下,$f>\frac{1}{2}$,$\varepsilon<\mu$,$f=\frac{1}{2}$,$\varepsilon=\mu$,$f<\frac{1}{2}$,$\varepsilon>\mu$,仅在$\varepsilon\in(\mu-kT,\mu+kT)$范围内,电子分布与$T=0$下显著差异,从而对热容有贡献;
能量在$\mu$附近$kT$范围内对热容有贡献的有效电子数:$N_{\text{有效}}\approx\frac{kT}{\mu}N$,将能量均分定理用于有效电子,每一有效电子贡献热容$\frac{3}{2}k$,\textbf{金属中自由电子热容的贡献}:$C_V=\frac{3}{2}Nk\left(\frac{kT}{\mu}\right)$,室温下$\frac{kT}{\mu}\ll1$\\
$N$和$U$均可写为$I=\int_0^{\infty}\frac{\eta(\varepsilon)d\varepsilon}{e^{(\varepsilon-\mu)/k\mu}+1}$,其中$\eta(\varepsilon)$分别$=C\varepsilon^{1/2}$,$C\varepsilon^{3/2}$,$C=\frac{4\pi V}{h^3}(2m)^{3/2}$,设$\varepsilon-\mu=kTz$,$I=\int_{-\mu/kT}^{\infty}\frac{\eta(\mu+kTz)}{e^z+1}kTdz=kT\int_0^{\mu/kT}\frac{\eta(\mu-kTz)}{e^{-z}+1}dz+kT\int_0^{\infty}\frac{\eta(\mu+kTz)}{e^z+1}dz=kT\int_0^{\mu/kT}\eta(\mu-kTz)dz-kT\int_0^{\mu/kT}\frac{\eta(\mu-kTz)}{e^z+1}dz+kT\int_0^{\infty}\frac{\eta(\mu+kTz)}{e^z+1}dz=\int_0^{\mu}\eta(\varepsilon)\varepsilon-kT\int_0^{\mu/kT}\frac{\eta(\mu-kTz)}{e^z+1}dz+kT\int_0^{\infty}\frac{\eta(\mu+kT)}{e^z+1}dz$,因$\mu/kT\gg1$,$I=\int_0^{\mu}\eta(\varepsilon)d\varepsilon+kT\int_0^{\infty}\frac{\eta(\mu+kTz)-\eta(\mu-kTz)}{e^z+1}dz$,分子泰勒关于$z$展开得$I=\int_0^{\mu}\eta(\varepsilon)d\varepsilon+2(kT)^2\eta'(\mu)\int_0^{\infty}\frac{zdz}{e^z+1}+\cdots=\int_0^{\mu}\eta(\varepsilon)d\varepsilon+\frac{\pi^2}{6}(kT)^2\eta'(\mu)+\cdots$,故$U=\frac{2}{5}C\mu^{5/2}\left[1+\frac{5\pi^2}{8}\left(\frac{kT}{\mu}\right)^2\right]$,$N=\frac{2}{3}C\mu^{3/2}\left[1+\frac{\pi^2}{8}\left(\frac{kT}{\mu}\right)^2\right]\Rightarrow\mu=\frac{\hbar^2}{2m}\left(\frac{3N\pi^2}{V}\right)^{2/3}\left[1+\frac{\pi^2}{8}\left(\frac{kT}{\mu}\right)^2\right]^{-2/3}$,代入$\mu_0$式且在第二项中用$\frac{kT}{\mu_0}$代替$\frac{kT}{\mu}$得$\mu=\mu_0\left[1+\frac{\pi^2}{8}\left(\frac{kT}{\mu_0}\right)^2\right]^{-2/3}\approx\mu_0\left[1-\frac{\pi^2}{12}\left(\frac{kT}{\mu_0}\right)^2\right]$;\textbf{内能}:$U=\frac{2}{5}C\mu_0^{5/2}\left[1-\frac{\pi}{12}\left(\frac{kT}{\mu_0}\right)^2\right]^{5/2}\left[1+\frac{5\pi^2}{8}\left(\frac{kT}{\mu_0}\right)^2\right]=\frac{3}{5}N\mu_0\left[1+\frac{5}{12}\pi^2\left(\frac{kT}{\mu_0}\right)^2\right]$;\textbf{电子贡献的热容}:$C_V=\left(\frac{\partial U}{\partial T}\right)_V=Nk^2\frac{\pi^2}{2}\left(\frac{kT}{\mu_0}\right)=\gamma_0T$;\textbf{低温下金属热容}:电子和离子振动贡献热容之和$C_V=\gamma T+AT^3$
\end{multicols}

\begin{multicols}{2}
\noindent\textbf{Chap9 系综理论}
\textbf{9.1 相空间 刘维尔定理}
\textbf{最概然统计方法}:适用于近独立粒子系统;\textbf{系综理论}:可处理粒子间有相互作用的系统\\
\textbf{统计系综}:一定宏观条件下,大量性质和结构完全相同的处于各种微观状态的各自独立的系统的集合\\
\textbf{系综理论的两点假设}:宏观量是相应微观量的时间平均,时间平均等价于系统平均;平衡孤立系的一切可能微观态出现概率相等\\
每个粒子自由度:$r$,$N$个全同粒子组成的系统自由度:$f=Nr$;任一时刻系统微观态由$f$个广义坐标$q_1,\cdots,q_f$和$f$个相应的广义动量$p_1,\cdots,p_f$确定,以该$2f$个变量为正交坐标构成的$2f$维空间为\textbf{相($\Gamma$)空间}\\
% 多种粒子组成的系统自由度:$f=\sum_iN_ir_i$;
系统微观态随$t$改变,代表点在相空间中移动,遵从\textbf{哈密顿正则方程}:$\dot{q}=\frac{\partial H}{\partial p_i},\dot{p}_i=\frac{\partial H}{\partial q_i}(i=1,\cdots,f)$\\
% ;当系统能量为确定值,代表点轨道在能量曲面$H(q,p)=E$上;当能量有确定范围,轨道在能壳$E\leq H(q,p)\leq E+\Delta E$内\\
系综内各系统代表点各从其初态出发沿正则方程规定的轨道运动,在相空间中形成由\textbf{统计分布函数}$\rho(q_1,\cdots,q_f;p_1,\cdots,p_f;t)$表示的分布,$t$时刻系统出现在体积元$d\Omega=dq_1\cdots dq_fdp_1\cdots dp_f$中的概率:$\rho(q,p,t)d\Omega$;此亦可表示此时$d\Omega$内代表点数\\
% \textbf{平衡条件}:当系统平衡,其宏观性质不随$t$变化,任一宏观量均非$t$的函数,则分布函数定非$t$的函数,$\rho=\rho(q,p)$\\
稳定系综按不同宏观条件分$3$类:\textbf{由孤立系统组成的微观正则系综},\textbf{由恒温封闭系统组成的正则系综},\textbf{由开放系统组成的巨正则系综}\\
\textbf{刘维尔定理}:$\frac{d\rho}{dt}=0$;证:$\rho(q_i+\dot{q}_idt,p_i+\dot{p}_idt,t+dt)=\rho+\frac{d\rho}{dt}dt$,$\frac{d\rho}{dt}=\frac{\partial\rho}{\partial t}+\sum_i\left[\frac{\partial\rho}{\partial q_i}\dot{q}_i+\frac{\partial\rho}{\partial p_i}\dot{p}_i\right]$,$d\Omega$以$2f$对平面$q_i,q_i+dq_i;p_i,p_i+dp_i(i=1,\cdots,f)$为边界,$t$时刻$d\Omega$内代表点数:$\rho d\Omega$,$t+dt$时刻代表点数:$\left(\rho+\frac{\partial\rho}{\partial t}dt\right)d\Omega$,$dt$内代表点数增加$\frac{\partial\rho}{dt}dtd\Omega$,$d\Omega$在平面$q_i$上的边界面积:$dA=dq_1\cdots dq_{i-1}dq_{i+1}\cdots dq_fdp_1\cdots dp_f$,$dt$内经$dA$入$d\Omega$代表点数:$\rho\dot{q}_idtdA$,经平面$q_i+dq_i$出$d\Omega$代表点数:$(\rho\dot{q}_i)_{q_i+dq_i}dtdA=\left[(\rho\dot{q}_i)_{q_i}+\frac{\partial}{\partial q_i}(\rho\dot{q}_i)dq_i\right]dtdA$,以上两式相减得$dt$内经一对平面$q_i$,$q_i+dq_i$净入$d\Omega$代表点数:$-\frac{\partial}{\partial q_i}(\rho\dot{q}_i)dq_idtdA=-\frac{\partial}{\partial q_i}(\rho\dot{q}_i)dtd\Omega$,同理经一对平面$p_i$和$p_i+dp_i$净入$d\Omega$代表点数:$-\frac{\partial}{\partial p_i}(\rho\dot{p}_i)dtd\Omega$,以上两式相加并对$i$求和得在$dt$内净入$d\Omega$代表点数:$\frac{\partial\rho}{\partial t}dtd\Omega=-\sum_i\left[\frac{\partial(\rho\dot{q}_i)}{\partial q_i}+\frac{\partial(\rho\dot{p}_i)}{\partial p_i}\right]dtd\Omega\Rightarrow\frac{\partial\rho}{\partial t}+\sum_i\left[\frac{\partial(\rho\dot{q}_i)}{\partial q_i}+\frac{\partial(\rho\dot{p}_i)}{\partial p_i}\right]=0$,由正则方程有$\frac{\partial\dot{q}_i}{\partial q_i}+\frac{\partial\dot{p}_i}{\partial p_i}=0$,从而$\frac{d\rho}{dt}=\frac{\partial\rho}{\partial t}+\sum_i\left[\frac{\partial\rho}{\partial q_i}\dot{q}_i+\frac{\partial\rho}{\partial p_i}\dot{p}_i\right]=0$,这说明若随一代表点在相空间中运动,其邻域内代表点密度恒定;\textbf{刘氏定理}(另一形式):$\frac{\partial\rho}{\partial t}=-\sum_i\left[\frac{\partial\rho}{\partial q_i}\frac{\partial H}{\partial p_i}-\frac{\partial\rho}{\partial p_i}\frac{\partial H}{\partial q_i}\right]$;若$\rho$仅为$H$的函数,$\frac{\partial\rho}{\partial t}=0$;因对变换$t\rightarrow-t$保持不变,刘氏定理可逆\\
\textbf{9.2 微正则系综}:孤立系统组成的系综,有确定$N$,$V$,$E$(更严谨地,$E\in(E,E+\Delta E)$,$\because$系统表面粒子数$\ll N$,$|\Delta E|\ll E$)\\
$t$时刻系统状态处于体积元$dqdp$中概率:$\rho(q,p,t)dqdp$,$\rho(q,p,t)$满足归一化条件$\int\rho dpdq=1$;设当系统处于$dqdp$,微观量$B$取值$B(q,p)$,则微观量$B$在所有可能微观状态上的平均值:$\overline{B}(t)=\int B(q,p)\rho(q,p,t)dqdp$即与$B$对应的宏观物理量\\
\textbf{系综理论的理解方式}:$t$时刻状态在$dqdp$内的系统数$\propto\rho(q,p,t)$,从系综中任取一系统,其处于$dqdp$内的概率:$\rho dqdp$;\textbf{系综平均值}:微观量$B$在统计系综上的平均值$\overline{B}(t)=\int B(q,p)\rho(q,p,t)dqdp$\\
量子理论中,$t$时刻系统处于状态$s$的概率:$\rho_s(t)$,$\rho_s(t)$满足归一化条件$\sum_s\rho_s(t)=1$;设微观量$B$在状态$s$上取值$B_s$,则$B$在所有可能微观态上的平均值:$\overline{B}(t)=\sum_s\rho_s(t)B_s$即与$B$对应的宏观物理量;系综理论的关键问题:确定分布函数$\rho$\\
\textbf{等概率原理(微正则分布)经典表达}:$E\sim E+\Delta E$内所有可能微观态概率密度相等,$\rho(q,p)=\text{const},E\leq H(q,p)\leq E+\Delta E$,$\rho(q,p)=0,H(q,p)<E,E+\Delta E<H(q,p)$;
\textbf{等概率原理的量子表达}:每个粒子自由度:$r$,$N$个全同粒子组成的系统相空间体积元:$h^{Nr}$,微观态数:$\Omega=\frac{1}{N!h^{Nr}}\int_{E\leq H(q,p)\leq E+\Delta E}dqdp$;任一微观态$s$出现概率:$\rho_s=\frac{1}{\Omega}$\\
% 多种粒子组成的系统微观态数:$\Omega=\frac{1}{\prod_lN_i!h^{N_ir_i}}\int_{E\leq H\leq E+\Delta E}dqdp$;
\textbf{最概然分布理论}:宏观量是微观量在最概然分布下的取值;\textbf{系综理论}:宏观量是微观量在给定宏观条件下所有可能微观态上的平均值;若相对涨落很小,概率密度分布必有很尖锐的极大,最概然和系综平均近似相等,因相对涨落$\sim1/N$,对宏观系统两种方法所得结果相同\\
\textbf{9.3 微正则系综的热力学公式}
\textbf{热力学平衡条件}:设孤立系统$A^{(0)}$包含子系统$A_1$和$A_2$,分别有($N_1$,$V_1$,$E_1$,$\Omega_1(N_1,V_1,E_1)$),($N_2$,$V_2$,$E_2$,$\Omega_2(N_2,V_2,E_2)$),两子系统仅能交,无物交和体积变化,$E^{(0)}=E_1+E_2$,$\Omega^{(0)}(E_1,E_2)=\Omega_1(E_1)\Omega_2(E_2)=\Omega_1(E_1)\Omega_2(E^{(0)}-E_1)$,对给定$E^{(0)}$,$\Omega^{(0)}$取决于$E_1$,即总微观态数取决于能量在两子系统间的分配,由等概率原理,平衡态下孤立系统所有可能微观态出现概率相等,当$E_1=\overline{E_1}$,$\Omega^{(0)}$极大,此即最概然能量分配,为一很尖锐的极大,其他能量分配出现的概率$\ll$最概然分配的概率,可视$\overline{E_1}$,$\overline{E_2}$为两子系统达热平衡时的内能,$\frac{\partial\Omega^{(0)}}{\partial E_1}=\frac{\partial\Omega_1(E_1)}{\partial E_1}\Omega_2(E_2)+\Omega_1(E_1)\frac{\partial\Omega_2(E_2)}{\partial E_2}\frac{\partial E_2}{\partial E_1}=\frac{\partial\Omega_1(E_1)}{\partial E_1}\Omega_2(E_2)-\Omega_1(E_1)\frac{\partial\Omega_2(E_2)}{\partial E_2}=0\Rightarrow\left(\frac{\partial\ln\Omega_1(E_1)}{\partial E_1}\right)_{N_1,V_1}=\left(\frac{\partial\ln\Omega_2(E_2)}{\partial E_2}\right)_{N_2,V_2}$,当子系统间达热平衡,$\beta=\left(\frac{\partial\ln\Omega(N,V,E)}{\partial E}\right)_{N,V}$必相等,\textbf{热平衡条件}:$\beta_1=\beta_2$;对照热力学中热平衡条件:$\left(\frac{\partial S_1}{\partial U_1}\right)_{N_1,V_1}=\left(\frac{\partial S_2}{\partial U_2}\right)_{N_2,V_2}=\frac{1}{T}$得$\beta=\frac{1}{kT}$和\textbf{熵}:$S=k\ln\Omega$(将最概然分布理论在近独立系统中的结论推广到了粒子间相互作用的系统);两子系统间还可物交和改变体积,$V^{(0)}=V_1+V_2$,$N^{(0)}=N_1+N_2$,从而$\left(\frac{\partial\ln\Omega}{\partial V_1}\right)_{N_1,E_1}=\left(\frac{\partial\ln\Omega_2}{\partial V_2}\right)_{N_2,E_2}$,$\left(\frac{\partial\ln\Omega_1}{\partial N_1}\right)_{E_1,V_1}=\left(\frac{\partial\ln\Omega_2}{\partial N_2}\right)_{E_2,V_2}$,定义$\gamma=\left(\frac{\partial\ln\Omega(N,V,E)}{\partial V}\right)_{N,E}$,$\alpha=\left(\frac{\partial\ln\Omega(N,V,E)}{\partial N}\right)_{V,E}$,\textbf{平衡条件}:$\gamma_1=\gamma_2$,$\alpha_1=\alpha_2$;$\ln\Omega$微分得$d\ln\Omega=\beta dE+\gamma dV+\alpha dN$,与开系热力学基本方程$dS=\frac{dU}{T}+\frac{p}{T}dV-\frac{\mu}{T}dN$比较得$\gamma=\frac{p}{kT}$,$\alpha=-\frac{\mu}{kT}$,从而\textbf{平衡条件}:$T_1=T_2$(\textbf{热学平衡}),$p_1=p_2$(\textbf{力学平衡}),$\mu_1=\mu_2$(\textbf{化学平衡});经典理想气体中粒子位置互不相关,$1$个分子处于$V$中,可能微观态数$\Omega(N,E,V)\propto V$,$N$个处于$V$中,可能微观状态数$\Omega(N,E,V)\propto V^N$,令$\Omega(N,E,V)=CV^N$,则$\gamma=\frac{p}{kT}=\frac{\partial\ln\Omega}{\partial V}=\frac{N}{V}$,比较理想气体物态方程得\textbf{玻尔兹曼常数}$k=\frac{R}{N_0}$\\
\textbf{用微正则分布求热力学函数的过程}:(1)求$\Omega(N,E,V)$;(2)$S(N,E,V)=k\ln\Omega(N,E,V)$;(3)$S=S(N,E,V)\Rightarrow E=E(S,N,V)$;(4)由开系热力学基本方程,$T(S,V,N)=\left(\frac{\partial E}{\partial S}\right)_{V,N}$,$p=-\left(\frac{\partial E}{\partial V}\right)_{S,N}$;(5)$T=T(S,V,N)\Rightarrow S=S(T,V,N)$;(6)$p=p(S,V,N)\Rightarrow p(T,V,N)$,$E=E(S,V,N)\Rightarrow E(T,V,N)$\\
设理想气体含$N$个单原子分子,哈密顿量:$H=\sum_{i=1}^{3N}\frac{p_i^2}{2m}$,则$\Omega=\frac{1}{N!h^{3N}}\int_{E\leq H(q,p)\leq E+\Delta E}dqdp$,系统能量$<E$的微观态数:$\Sigma(E)=\frac{1}{N!h^{3N}}\int_{H(q,p)\leq E}dqdp=\frac{V^N}{N!h^{3N}}\int_{H(q,p)\leq E}dp$,做变换$p_i=\sqrt{2mE}x_i$得$\Sigma(E)=K\frac{V^N}{N!h^{3N}}(2mE)^{\frac{3N}{2}}$,其中$K=\idotsint_{\sum_ix_i^2\leq 1}dx_1\cdots dx_{3N}=\frac{\pi^{3N/2}}{\left(\frac{3N}{2}\right)!}$等于$3N$维空间中半径为$1$的球体积,故$\Sigma(E)=\left(\frac{V}{h^3}\right)^N\frac{(2\pi mE)^{3N/2}}{N!\left(\frac{3N}{2}\right)!}$,能壳$E\sim E+\Delta E$内微观态数:$\Omega(E)=\frac{\partial\Sigma}{\partial E}\Delta E=\frac{3N}{2}\frac{\Delta E}{E}\Sigma(E)$,$S=k\ln\Omega=Nk\ln\left[\frac{V}{h^3N}\left(\frac{4\pi mE}{3N}\right)^{3/2}\right]+\frac{5}{2}Nk+k\left[\ln\left(\frac{3N}{2}\right)+\ln\left(\frac{\Delta E}{E}\right)\right]$,因$\lim_{N\rightarrow\infty}\frac{\ln N}{N}=0$,$S=Nk\ln\left[\frac{V}{h^3N}\left(\frac{4\pi mE}{3N}\right)^{3/2}\right]+\frac{5}{2}Nk$,$E(N,S,V)=\frac{3h^2N^{5/3}}{4\pi mV^{2/3}}e^{\frac{2S}{3Nk}-\frac{5}{3}}$,$T=\left(\frac{\partial E}{\partial S}\right)_{N,V}=\frac{2}{3Nk}E$,$E=\frac{3}{2}NkT$,$p=-\left(\frac{\partial E}{\partial V}\right)_{N,S}=\frac{2}{3}\frac{E}{V}$,以上两式联立得$pV=NkT$,$S=Nk\ln\left[\frac{V}{N}\left(\frac{2\pi mkT}{h^2}\right)^{3/2}\right]+\frac{5}{2}Nk$\\
\textbf{9.4 正则系综}
\textbf{正则分布}:有确定$N$,$V$,$T$的系统的分布函数\\
设有确定$N$,$V$,$T$的系统与大热源接触,热源很大,能交不改变热源温度,$E^{(0)}=E+E_r$,$E\ll E^{(0)}$,平衡后两者温度相等,平衡态下每个可能微观状态出现概率相等,故系统处于状态$s$的概率$\rho_s\propto\Omega_r(E^{(0)}-E_s)$,因$\frac{E_s}{E^{(0)}}\ll1$,将$\ln\Omega_r$展开为$E_s$的幂级数并保留$1$阶项,$\ln\Omega_r(E^{(0)}-E_s)=\ln\Omega_r(E^{(0)})+\left(\frac{\partial\ln\Omega_r}{\partial E_r}\right)_{E_r=E^{(0)}}(-E_s)=\ln\Omega_r(E^{(0)})-\beta E_s$,其中$\beta=\left(\frac{\partial\ln\Omega_r}{\partial E_r}\right)_{E_r=E^{(0)}}=\frac{1}{kT}$,$T$--热源温度,因达热平衡,亦系统温度,故$\rho_s\propto e^{-\beta E_s}$,归一化得$\rho_s=\frac{1}{Z}e^{-\beta E_s}$,其中\textbf{配分函数}:$Z=\sum_se^{-\beta E_s}$,$\sum_s$为对($N$,$V$)的系统的所有微观态求和,这说明系统处于微观态$s$的概率仅与$E_s$有关;若$E_l$表示系统各个能级,$\Omega_l$表示其简并度,则系统处于$E_l$的概率:$\rho_l=\frac{1}{Z}\Omega_le^{-\beta E_l}$,其中\textbf{配分函数}:$Z=\sum_l\Omega_le^{-\beta E_l}$,$\sum_l$是在给定$N$,$V$下对系统所有能级求和;\textbf{正则分布的经典表达式}:$\rho(q,p)dqdp=\frac{1}{N!h^{Nr}}\frac{e^{-\beta E(q,p)}}{Z}dqdp$,其中\textbf{配分函数}:$Z=\frac{1}{N!h^{Nr}}\int e^{-\beta E(q,p)}dqdp$\\
\textbf{9.5 正则系综理论的热力学公式}
\textbf{内能}:$U=\overline{E}=\frac{1}{Z}\sum_sE_se^{-\beta E_s}=\frac{1}{Z}\left(-\frac{\partial}{\partial\beta}\right)\sum_se^{-\beta E_s}=-\frac{\partial}{\partial\beta}\ln Z$\\
\textbf{广义力}:$Y=\frac{1}{Z}\sum_s\frac{\partial E_s}{\partial y}e^{-\beta E_s}=\frac{1}{Z}\left(-\frac{1}{\beta}\frac{\partial}{\partial y}\right)\sum_se^{-\beta E_s}=-\frac{1}{\beta}\frac{\partial}{\partial y}\ln Z$;\textbf{压强}:$p=\frac{1}{\beta}\frac{\partial}{\partial V}\ln Z$\\
$\beta(dU-Ydy)=-\beta d\left(\frac{\partial}{\partial\beta}\ln Z\right)+\frac{\partial}{\partial y}\ln Zdy$,因$Z$为$\beta,y$的函数,$\ln Z$的全微分为$d\ln Z=\frac{\partial}{\partial\beta}\ln Zd\beta+\frac{\partial}{\partial y}\ln Zdy$,代入前式得$\beta(dU-Ydy)=d\left(\ln Z-\beta\frac{\partial}{\partial\beta}\ln Z\right)$,这说明$\beta$为$dU-Ydy$的积分因子,与热力学公式$\frac{1}{T}(dU-Ydy)=dS$比较得$\beta=\frac{1}{kT}$,$S=k\left(\ln Z-\beta\frac{\partial}{\partial\beta}\ln Z\right)$;\textbf{自由能}:$F=U-TS=-kT\ln Z$\\
\textbf{能量涨落}:统计系综中各系统能量与其平均值的偏差的平方的平均值,$\overline{(E-\overline{E})^2}=\sum_s\rho_s(E_s-\overline{E})^2=\sum_s\rho_s\left[E_s^2-2\overline{E}E_s+(\overline{E})^2\right]=\sum_s\rho_sE_s^2-2\overline{E}\sum_s\rho_sE_s+(\overline{E})^2\sum_s\rho_s=\overline{E^2}-(\overline{E})^2$;对正则分布,$\frac{\partial\overline{E}}{\partial\beta}=\frac{\partial}{\partial\beta}\frac{\sum_sE_se^{-\beta E_s}}{\sum_se^{-\beta E_s}}=-\frac{\sum_sE_s^2e^{-\beta E_s}}{\sum_se^{-\beta E}}+\frac{\left(\sum_sE_se^{-\beta E_s}\right)^2}{\left(\sum_se^{-\beta E_s}\right)^2}=-\left[\overline{E^2}-(\overline{E})^2\right]$,故$\overline{(E-\overline{E})^2}=-\frac{\partial\overline{E}}{\partial\beta}=kT^2\frac{\partial\overline{E}}{\partial T}=kT^2C_V$,\textbf{能量相对涨落}:$\frac{\overline{(E-\overline{E})^2}}{(\overline{E})^2}=\frac{kT^2C_V}{(\overline{E})^2}$,这说明$C_V>0$;其中广延量$\overline{E}$,$C_V\propto N$,故相对涨落反比$N$,对单原子分子理想气体,$\overline{E}=\frac{3}{2}NkT$,$C_V=\frac{3}{2}Nk$,$\frac{\overline{(E-\overline{E})^2}}{(\overline{E})^2}=\frac{2}{3N}$,故宏观系统($N\sim10^{23}$)可忽略能量相对涨落\\
\textbf{9.6 实际气体的物态方程}
(分子间作用不能忽略)
推导方法:(1)建立实际气体微观模型;(2)系统能量表达式;(3)配分函数;(4)物方\\
单原子分子气体能量:$E=\sum_{k=1}^{3N}\frac{p_k^2}{2m}+\sum_{i<j}\varphi(r_{ij})$,\textbf{配分函数}:$Z=\frac{1}{N!h^{3N}}\int e^{-\beta E}dqdp=\frac{1}{N!h^{3N}}\int e^{-\beta\sum_{k=1}^{3N}\frac{p_k^2}{2m}}dp\int e^{-\beta\sum_{i<j}\varphi(r_{ij})}dq=\frac{1}{N!h^{3N}}AQ$,其中$A=\prod_{k=1}^{3N}\int_{-\infty}^{+\infty}e^{-\beta\frac{p_k^2}{2m}}dp_k=\left(\frac{2\pi m}{\beta}\right)^{3N/2}$,$Q=\int e^{-\beta\sum_{i<j}\varphi(r_{ij})}dq=\idotsint e^{-\beta\sum_{i<j}\varphi(r_{ij})}d\bm{\tau}_1\cdots d\bm{\tau}_N$,\textbf{配分函数}:$Z=\frac{1}{N!}\left(\frac{2\pi m}{\beta h^2}\right)^{3N/2}Q$,对每对分子引入函数$f_{ij}=e^{-\beta\varphi(r_{ij})}-1$,当$r_{ij}>$互作用力程($\sim10^{-10\sim-9}$m),$\varphi(r_{ij})=0$,$f_{ij}=0$,当$r_{ij}<$力程,$f_{ij}\neq0$,集团展开:$Q=\int\prod_{i<j}(1+f_{ij})d\bm{\tau}=\int(1+\sum_{i<j}f_{ij}+\sum_{i<j}\sum_{i'<j'}f_{ij}f_{i'j'}+\cdots)d\bm{\tau}$,若仅保留首项,即不计分子互作用,$Q=V^N$,近似为理想气体,第二项仅当$1$对分子($i$,$j$)在力程内才$\neq0$,第三项仅当$2$对分子($i$,$j$),($i'$,$j'$)均在力程内才$\neq0$,对稀薄气体,$2$个以上分子同时相碰的概率极小,故仅保留前两项:$Q=\int(1+\sum_{i<j}f_{ij})\bm{\tau}=V^N+\int\sum_{i<j}f_{ij}d\bm{\tau}$,第二项化为$\int\sum_{i<j}f_{ij}\bm{\tau}=\sum_{i<j}\int f_{ij}d\bm{\tau}=\sum_{i<j}V^{N-2}\iint f_{ij}d\bm{\tau}_id\bm{\tau}_j=N^2V^{N-2}\iint f_{12}d\bm{\tau}_1\bm{\tau}_2$,引入相对坐标$\bm{r}=\bm{r}_2-\bm{r}_1$,质心坐标$\bm{r}_C=\frac{m_1\bm{r}_1+m_2\bm{r}_2}{m_1+m_2}$,从而$\iint f_{12}d\bm{\tau}_1d\bm{\tau}_2=\iint f_{12}\bm{r}d\bm{r}_C=V\int f_{12}d\bm{r}$,$Q=V^N\left[1+\frac{N^2}{2V}\int f_{12}d\bm{r}\right]$,$\ln Q=N\ln V+\ln\left[1+\frac{N^2}{2V}\int f_{12}d\bm{r}\right]$,假设$\frac{N^2}{2V}\int f_{12}d\bm{r}\ll1$,$\ln Q=N\ln V+\frac{N^2}{2V}\int f_{12}d\bm{r}$,\textbf{物态方程}:$p=\frac{1}{\beta}\frac{\partial}{\partial V}\ln Z=\frac{1}{\beta}\frac{\partial}{\partial V}\ln Q=\frac{1}{\beta}\frac{N}{V}\left[1-\frac{N}{2V}\int f_{12}d\bm{r}\right]$,比较二阶近似的昂尼斯方程:$pV=NkT\left[1+\frac{nB}{V}\right]$得第$2$位力系数$B=-\frac{N_A}{2}\int f_{12}d\bm{r}$;\textbf{纳德·琼斯半经验公式}:两分子互作用势:$\phi(r)=\phi_0\left[\left(\frac{r_0}{r}\right)^{12}-2\left(\frac{r_0}{r}\right)^6\right]$,近似为$\phi(r)=+\infty$,$r<r_0$,$\phi(r)=-\phi_0\left(\frac{r_0}{r}\right)^6$,故$B=-\frac{N_A}{2}\int(e^{-\frac{\phi(r)}{kT}}-1)r^2\sin\theta drd\theta d\phi=-2\pi N_A\int_0^{\infty}(e^{-\frac{\phi(r)}{kT}}-1)r^2dr=2\pi N\left[\int_0^{r_0}r^2dr-\int_{r_0}^{\infty}(e^{-\frac{\phi(r)}{kT}}-1)r^2dr\right]$,若$T$足够高,分子热激发$>$互作用势,$e^{-\frac{\phi}{kT}}=1-\frac{\phi}{kT}$,$B=2\pi N_A\left[\frac{r_0^3}{3}-\frac{\phi_0r_0^3}{3kT}\right]$,令$B=b-\frac{a}{N_AkT}$,其中$b=\frac{2\pi}{3}N_Ar_0^3=4N_Av=4V_{0m}$,$a=\frac{2\pi}{3}N_A^2\phi_0r_0^3=4N_A\phi_0V_{0m}$与分子引力有关,\textbf{物态方程}近似为$pV=NkT\left[1+\frac{nb}{V}\right]-\frac{n^2a}{V}$或$\left(p+\frac{n^2a}{V^2}\right)\frac{V}{1+\frac{nb}{V}}=NkT$,因$\frac{nb}{V}\ll1$,$\frac{1}{1+\frac{nb}{V}}=1-\frac{nb}{V}$,物态方程近似为$\left(p+\frac{n^2a}{V}\right)(V-nb)=NkT$\\
\textbf{9.7 固体的热容}
固体原子间相互作用很强,各原子在一定平衡位置附近做三维非简谐微振动\\
系统能量由原子振动动能$\sum_{i=1}^{3N}\frac{p_{\xi_i}^2}{2m}$和势能(原子偏离平衡位置位移的幂级数,近似至二阶)$\phi=\phi_0+\sum_i\left(\frac{\partial\phi}{\partial\xi_i}\right)_0\xi_i+\\\frac{1}{2}\sum_{i,j}\left(\frac{\partial^2\phi}{\partial\xi_i\partial\xi_j}\right)\xi_i\xi_j$组成,因所有原子均处于平衡位置附近,$\left(\frac{\partial\phi}{\partial\xi_i}\right)_0=0$,令$a_{ij}=\left(\frac{\partial^2\phi}{\partial\xi_i\partial\xi_j}\right)_0$,$E=\sum_{i=1}^{3N}\frac{p_{\xi_i}^2}{2m}+\frac{1}{2}\sum_{i,j}a_{ij}\xi_i\xi_j+\varphi_0$,用线性变换将各$\xi_i$组合为\textbf{简正坐标}$q_i$,上式化为平方和:$E=\frac{1}{2}\sum_{i=1}^{3N}(p_i+\omega_i^2q_i^2)+\phi_0$,这$3N$个简正坐标的运动是相互独立的\textbf{简正振动},特征频率:$\omega_i$,由量子理论,$3N$个谐振子的能量:$E=\phi_0+\sum_{i=1}^{3N}\hbar\omega_i(n_i+\frac{1}{2})$,\textbf{配分函数}:$Z=\sum_se^{-\beta E_s}=e^{-\beta\phi_0}\sum_{\{n_i\}}e^{-\beta\sum_i\hbar\omega_i(n_i+\frac{1}{2})}=e^{-\beta\phi_0}\sum_{\{n_i\}}\prod_ie^{-\beta\hbar\omega_i(n_i+\frac{1}{2})}=e^{-\beta\phi_0}\prod_i\sum_{n_i=0}^{\infty}e^{-\beta\hbar\omega_i(n_i+\frac{1}{2})}=e^{-\beta\phi_0}\prod_i\frac{e^{-\beta\hbar\omega_i/2}}{1-e^{-\beta\hbar\omega_i}}$,\textbf{内能}:$U=-\frac{\partial}{\partial\beta}\ln Z=U_0+\sum_{i=1}^{3N}\frac{\hbar\omega_i}{e^{\beta\hbar\omega_i}-1}$,其中\textbf{结合能}:$U_0=\phi_0+\sum_{i=1}^{3N}\frac{\hbar\omega_i}{2}$,\textbf{零点能}:$\sum_{i=1}^{3N}\frac{\hbar\omega_i}{2}$,\textbf{热运动能量}:$\sum_{i=1}^{3N}\frac{\hbar\omega_i}{e^{\beta\hbar\omega_i}-1}$,需简正振动的频率分布(振动频谱),才能求出内能;\textbf{爱因斯坦模型}:假设固体中所有原子均独立地做频率相同的简谐运动,$U=3N\frac{\hbar\omega}{2}+\frac{3N\hbar\omega}{e^{\beta\hbar\omega}-1}$;\textbf{德拜固体理论}:视固体为连续弹性媒介,$3N$个简正振动是其基本波动,对一定的波矢$k$,固体中的波动含仅一种振动方式的纵波和两种振动方式(两个正交方向上的偏振)的横波,可用波矢和偏振方向标志$3N$个简正振动状态,两种波的色散关系:$\omega=c_lk$,$\omega=c_tk$,$V$中$\omega\sim\omega+d\omega$范围内简正振动数:$D(\omega)d\omega=\frac{V}{2\pi^2}\left(\frac{1}{c_l^3}+\frac{2}{c_t^3}\right)\omega^2d\omega$,设$B=\frac{V}{2\pi^2}\left(\frac{1}{c_l^3}+\frac{2}{c_t^3}\right)$,最大圆频率$\omega_D$,\textbf{德拜频谱}:$D(\omega)d\omega=B\omega^2d\omega$,$\omega\leq\omega_D$,$D(\omega)d\omega=0$,$\omega\geq\omega_D$,$\int_0^{\omega_D}B\omega^2d\omega=3N\Rightarrow\omega_D=\frac{9N}{B}$,\textbf{内能}:$U=U_0+\int_0^{\omega}D(\omega)\frac{\hbar\omega}{e^{\beta\hbar\omega}-1}d\omega=U_0+B\int_0^{\omega_D}D(\omega)\frac{\hbar\omega^3}{e^{\frac{\hbar\omega}{kT}}-1}$,设$y=\frac{\hbar\omega}{kT}$,$x=\frac{\hbar\omega_D}{kT}=\frac{\theta_D}{T}$,其中$\theta_D$--\textbf{德拜特征温度},是物体的特征参量,\textbf{内能}:$U=U_0+9NkT\left(\frac{kT}{\hbar\omega_D}\right)^3\int_0^x\frac{y^3dy}{e^y-1}$,定义\textbf{德拜函数}:$\mathscr{D}(x)=\frac{3}{x^3}\int_0^x\frac{y^3dy}{e^y-1}$,\textbf{内能}:$U=U_0+3NkT\mathscr{D}(x)$;高温下,$T\gg\theta_D$,$y\ll1$,$e^y=1+y$,$\mathscr{D}(x)\approx\frac{3}{x^3}\int_0^xy^2dy=1$,\textbf{内能}:$U=U_0+3NkT$,热容:$C_V=3Nk$,与经典统计理论结果一致;低温下,$T\ll\theta_D$,$x\gg1$,$\mathscr{D}(x)\approx\frac{3}{x^3}\int_0^{\infty}\frac{y^3dy}{e^y-1}=\frac{\pi^4}{5x^3}$,\textbf{内能}:$U=U_0+3Nk\frac{\pi^4}{5}\frac{T^4}{\theta_D^3}$,$C_V=3Nk\frac{4\pi^4}{5}\left(\frac{T}{\theta_D}\right)^3$(\textbf{德拜$T^3$律});符合非金属和$T\geq3$K下金属固体,$T\leq3$K下金属固体不能忽略电子对热容的影响,德拜仅考虑其原子部分,且视固体为连续弹性介质,忽略了其中原子的离散结构;对$\lambda\gg a$(--原子间平均距离)的简正振动,与实际情况近似,对$\lambda\sim a$,离散结构不能忽略\\
有某一波矢和偏振的简正振动处在量子数为$n$的激发态,相当于有某一准动量和偏振的声子,有不同波矢和偏振的简正振动,相当于状态不同的声子,因简正振动的量子数可取任意非负整数,任一状态的声子数是任意的,声子遵从玻色分布,平衡态下简正振动的能量不断变化,相当于不同状态的声子不断产生和湮灭,声子数不守恒,声子气体化学势$=0$,由玻色分布$U=U_0+\sum_{i=1}^{3N}\frac{\hbar\omega_i}{e^{\beta\hbar\omega_i}-1}$\\
\textbf{9.8 液$^4$He的性质和朗道超流理论}
$^3$He--费米子,$^4$He--玻色子,He原子间相互作用很弱,原子质量很小,故零点振动能很大,常压下接近$0$K时仍可保持液态,此时量子相应主导,液He为量子液体\\
\textbf{$\lambda$相变}:正常相$^4$He I沿饱和蒸气压曲线降温,在$T_{\lambda}=2.18$K和比容$v_{\lambda}=46.2\AA^3/$atom处相变为He II,相变处无潜热/体积变化,比热以对数形式$\rightarrow+\infty$,为二级相变,比热线像$\lambda$,故名之,$T^3$附近,比热以$T^3$形式$\rightarrow0$\\
\textbf{液He II的特性}:1.\textbf{超流性}:能沿极细的毛细管流动而无粘滞性,\textbf{临界速度}之上,超流性破坏;2.用细丝悬薄圆盘浸入并使盘做扭转振动,测得粘滞系数与正常相相似,比毛细法所得大$10^6$倍,强烈依赖于温度,$\rightarrow0$随$T\rightarrow0$K;3.\textbf{力热效应}:从容器A经多孔塞或极细毛细管流出时,A内余液$T\uparrow$,其逆过程称\textbf{热力效应};4.热导率很大,为室温下Cu800倍,不似普通流体$\propto$温度梯度\\
\textbf{二流体模型}:a.液He II由正常流体和超流体组成,超流体无粘滞性和熵,正常流体有,$\rho_s$,$\rho_n$--超/正常流体质量密度,$v_s$,$v_n$--两者速度场,总质量密度$\rho=\rho_s+\rho_n$,总质量流$\rho\bm{v}=\rho_s\bm{v}_s+\rho_n\bm{v}_n$;b.$T=0$K时,超流体$100\%$,$T\geq T_{\lambda}$,正常流体$100\%$,$0<T<T_{\lambda}$,$\rho_s/\rho$为温度的函数;c.超流体速度场无旋,$\nabla\times\bm{v}_s=0$,两种流体可相互流动而彼此间无摩擦(动量交换);超流体可通过毛细管,故1.;仅正常流体对盘有阻尼,故2.;仅超流体流出,不带走熵,故A内$S\uparrow\Rightarrow T\uparrow$,故3.;若$T$均匀的液He II中某点$T$略$\uparrow$,热点密度涨落,$\rho_n/\rho\uparrow$,$\rho_s/\rho\downarrow$,为恢复平衡,附近超流体流向热点,正常流体流离,称\textbf{内运流},此过程很快,故4.\\
因有两种成分,朗道预言He II中有两种独立的振动波:1.若$\bm{v}_s$,$\bm{v}_n$同向,则振动波传递密度和压强的变化,为普通的声(\textbf{第一声});2.若$\bm{v}_s$,$\bm{v}_n$反向,则可能在保持$\rho$基本不变的情况下,$\rho_s$,$\rho_n$分别涨落,因超流成分无熵,$\rho_n$的涨落决定了$s$和$T$的涨落;将液He II视为受弱激发的量子玻色系统,弱激发态与基态($T=0$K)的偏离表现为平静的背景上出现由元激发或准粒子组成的气体,前者对应超流体,后者对应正常流体;当$T$很低,元激发密度很低,可视为元激发的理想气体,$p$,$\varepsilon(p)$--元激发的动量/能量,$n(p)$--元激发数,系统低激发态的总能量:$E=E_0+\sum_pn(\bm{p})\varepsilon(\bm{p})$,总动量:$\bm{P}=\sum_{\bm{p}}\bm{p}n(\bm{p})$;再假设液He II中有两种不同的玻色元激发--声子/旋子,1.实验发现当$T\ll T_{\lambda}$,比热随$T^3$变化,此为声子特征,能谱:$\varepsilon=cp$,$c$--声子速度,2.实验发现当温度稍高,比热有一如$\exp[-\Delta/k_BT]$的附加项,$\Delta=$const,故推测对较大动量,元激发能量有一$\Delta$能隙,该动量范围内能谱:$\varepsilon(p)=\hbar\omega_k=\Delta+\frac{(p-p_0)^2}{2m^*}=\Delta+\frac{\hbar^2(k-k_0)^2}{2m^*}$,$m^*$--旋子有效质量;准粒子在能量$\hbar\omega_k$的平均占据数:$\langle n_k\rangle=1/(e^{\beta\hbar\omega_k}-1)$,内能:$U=E_0+\sum_k\hbar\omega_k\langle n_k\rangle=E_0+\frac{V}{2\pi^2}\int_0^{\infty}\frac{k^2\hbar\omega_kdk}{e^{\beta\hbar\omega_k}-1}$,定容比热:$C_V=\left(\frac{\partial U}{\partial T}\right)_V$,声子贡献比热:$\frac{C_{\text{phonon}}}{Nk_B}=\frac{2\pi^2v(k_BT)^3}{15(\hbar c)^3}$,当$k_BT/\Delta$较小,旋子贡献比热$\frac{C_{\text{roton}}}{Nk_B}=\frac{2\sqrt{m^*}(k_0\Delta)^2ve^{-\Delta/k_BT}}{\hbar(2\pi k_BT)^{3/2}}$,这些结果与实验符合得很好\\
\textbf{稀释制冷}:混合室内$^3$He-$^4$He两相共存,正常相$^3$He于上,$^3$He的$^4$He超流稀溶液于下,不断抽走蒸发室的超流相中的$^3$He蒸汽,混合室中液$^3$He不断溶入超流相,此为$S\uparrow$过程,故吸热$Q=T\Delta S$\\
\textbf{9.10 巨正则分布系综}
\textbf{巨正则分布}:有确定$V$,$T$,$\mu$的系统(与环境平衡的开系)的分布函数\\
系统与原组成复合孤立系统,复合系统有确定$N^{(0)}=N+N_r$,$E^{(0)}=E+E_r$,因源很大,有$E\ll E^{(0)}$,$N\ll N^{(0)}$;当系统处于$N$,$E_s$的微观态$s$,源可处于$N^{(0)}-N$,$E^{(0)}-E_s$的任一微观态,源的微观态数:$\Omega_r(N^{(0)}-N,E^{(0)}-E_s)$,此亦复合系统可能微观状态数,由等概率原理,系统处于$N$,$E_s$的微观态$s$的概率:$\rho_{Ns}\propto\Omega_r(N^{(0)},E^{(0)}-E_s)$;将$\ln\Omega_r$在$N^{(0)}$,$E^{(0)}$处展开为$N$,$E_s$的幂级数,取到$1$阶项,$\ln\Omega_r(N^{(0)}-N,E^{(0)}-E)=\ln\Omega_r(N^{(0)},E^{(0)})+\left(\frac{\partial\ln\Omega}{\partial N_r}\right)_{N_r=N^{(0)}}(-N)+\left(\frac{\partial\ln\Omega_r}{\partial E_r}\right)_{E_r=E^{(0)}}(-E_s)=\ln\Omega_r(N^{(0)},E^{(0)})-\alpha N-\beta E_s$,其中$\alpha=\left(\frac{\partial\ln\Omega_r}{\partial N_r}\right)_{N_r=N^{(0)}}=-\frac{\mu}{kT}$,$\beta=\left(\frac{\partial\ln\Omega_r}{\partial E_r}\right)_{E_r=E^{(0)}}=\frac{1}{kT}$,因$\ln\Omega_r(N^{(0)},E^{(0)})=$const,$\rho_{Ns}\propto e^{-\alpha N-\beta E_s}$,归一化得$\rho_{Ns}=\frac{1}{\Xi}e^{-\alpha N-\beta E_s}$,其中巨配分函数:$\Xi=\sum_{N=0}^{\infty}\sum_se^{-\alpha N-\beta E_s}$,\textbf{巨正则分布的经典表达式}:$\rho_Ndqdp=\frac{1}{N!h^{Nr}}\frac{e^{-\alpha N-\beta E(q,p)}}{\Xi}d\Omega$,其中\textbf{巨配分函数}:$\Xi=\sum_N\frac{e^{-\alpha N}}{N!h^{Nr}}\int e^{-\beta E(q,p)}d\Omega$\\
\textbf{9.11 巨正则系综理论的热力学公式}~系统\textbf{平均粒子数}:给定$V$,$T$,$\mu$下所有可能微观态粒子数的平均值,$\overline{N}=\sum_N\sum_sN\rho_{Ns}=\frac{1}{\Xi}\sum_N\sum_sNe^{-\alpha N-\beta E_s}=\frac{1}{\Xi}\left(-\frac{\partial}{\partial\alpha}\right)\sum_N\sum_se^{-\alpha N-\beta E_s}=\frac{1}{\Xi}\left(-\frac{\partial}{\partial\alpha}\right)\Xi=-\frac{\partial}{\partial\alpha}\ln\Xi$\\
\textbf{内能}:$U=\overline{E}=\frac{1}{\Xi}\sum_N\sum_sE_se^{-\alpha N-\beta E_s}=\frac{1}{\Xi}\left(-\frac{\partial}{\partial\beta}\right)\Xi=-\frac{\partial}{\partial\beta}\ln\Xi$\\
\textbf{广义力}:$Y=\frac{1}{\Xi}\sum_N\sum_s\frac{\partial E_s}{\partial y}e^{-\alpha N-\beta E_s}=\frac{1}{\Xi}\left(-\frac{1}{\beta}\frac{\partial}{\partial y}\right)\Xi=-\frac{1}{\beta}\frac{\partial}{\partial y}\ln\Xi$;\textbf{压强}:$p=\frac{1}{\beta}\frac{\partial}{\partial\beta}\ln\Xi$\\
$\beta(dU-Ydy+\frac{\alpha}{\beta}d\overline{N})=-\beta d\left(\frac{\partial\ln\Xi}{\partial\beta}\right)+\frac{\partial\ln\Xi}{\partial y}dy-\alpha d\left(\frac{\partial}{\partial\alpha}\ln\Xi\right)$,因$\ln\Xi$为$\alpha$,$\beta$,$y$的函数,其全微分为$d\ln\Xi=\frac{\partial\ln\Xi}{\partial\beta}d\beta+\frac{\partial\ln\Xi}{\partial\alpha}d\alpha+\frac{\partial\ln\Xi}{\partial y}dy$,故$\beta\left(dU-Ydy+\frac{\alpha}{\beta}d\overline{N}\right)=d\left(\ln\Xi-\alpha\frac{\partial\Xi}{\partial\alpha}-\beta\frac{\partial\ln\Xi}{\partial\beta}\right)$,开系热力学基本方程$\frac{1}{T}(dU-Ydy-\mu d\overline{N})=dS$说明$1/T$为$\left(dU-Ydy+\mu d\overline{N}\right)$的积分因子,故$\beta=\frac{1}{kT}$,$\alpha=-\frac{\mu}{kT}$,$S=k\left(\ln\Xi-\alpha\frac{\partial\ln\Xi}{\partial\alpha}-\beta\frac{\partial\ln\Xi}{\partial\beta}\right)$\\
\textbf{粒子数涨落}:$\overline{(N-\overline{N})^2}=\overline{N^2}-\overline{N}^2$,因$\overline{N^2}=\frac{1}{\Xi}\sum_N\sum_sN^2e^{-\alpha N-\beta E_s}=\frac{1}{\Xi}\left(-\frac{\partial}{\partial\alpha}\right)\sum_N\sum_sNe^{-\alpha N-\beta E_s}=\frac{1}{\Xi}\left[\frac{\partial}{\partial\alpha}(\overline{N}\Xi)\right]=-\frac{1}{\Xi}\left[\Xi\frac{\partial\overline{N}}{\partial\alpha}+\overline{N}\frac{\partial\Xi}{\partial\alpha}\right]=-\frac{\partial\overline{N}}{\partial\alpha}+\overline{N}^2$,故$\overline{(N-\overline{N})^2}=-\left(\frac{\partial\overline{N}}{\partial\alpha}\right)_{\beta,y}=kT\left(\frac{\partial\overline{N}}{\partial\mu}\right)_{T,V}$;\textbf{粒子数相对涨落}:$\frac{\overline{(N-\overline{N})^2}}{\overline{N}^2}=\frac{kT}{\overline{N}^2}\left(\frac{\partial\overline{N}}{\partial\mu}\right)_{T,V}$;由$dG=-SdT+Vdp+\mu dn$有$d(\overline{N}\mu)=-(\overline{N}s)dT+(\overline{N}v)dp+\mu d\overline{N}\Rightarrow d\mu=vdp-sdT$,因$d\mu=\frac{\partial\mu}{\partial v}dv+\frac{\partial u}{\partial T}dT$,故$\left(\frac{\partial\mu}{\partial v}\right)_T=v\left(\frac{\partial p}{\partial v}\right)_T$,注意$v=\frac{V}{\overline{N}}$,当$V$不变而$\overline{N}$变,$\left(\frac{\partial\mu}{\partial v}\right)_T=\left(\frac{\partial\mu}{\partial\overline{N}}\right)\left(\frac{\partial\overline{N}}{\partial v}\right)_{T,V}=-\frac{\overline{N}^2}{V}\left(\frac{\partial\mu}{\partial\overline{N}}\right)_{T,V}$,故\textbf{粒子数相对涨落}:$\frac{\overline{(N-\overline{N})^2}}{\overline{N}^2}=\frac{kT}{\overline{N}^2}\left(\frac{\partial\overline{N}}{\partial\mu}\right)_{T,V}=-\frac{kT}{Vv}\left(\frac{\partial v}{\partial p}\right)_T=\frac{kT}{V}\kappa_T$;因广延量$V\propto\overline{N}$,当$\kappa_T$有限,相对涨落反比$\overline{N}$,故宏观系统涨落很小\\
用巨正则分布求热力学量相当于选自变量为$\mu$,$V$,$T$的巨热力学势$J$为特性函数;正则分布相当于选$F(N,V,T)$\\
\textbf{9.12 巨正则系综理论的简单应用}
\textbf{吸附现象}:表面有$N_0$个吸附中心,各可吸附一气体分子,分子吸附后能量为$-\varepsilon_0$,视气体为热源和粒子源,被吸附分子组成与之能交和物交的系统,遵从巨正则分布,当$N$个分子吸附,系统能量:$-N\varepsilon_0$,巨配分函数:$\Xi=\sum_{N=0}^{N_0}\sum_se^{-\alpha N-\beta E_s}=\sum_{N=0}^{N_0}e^{\beta(\mu+\varepsilon_0)N}\frac{N_0!}{N!(N_0-N)!}=[1+e^{\beta(\mu+\varepsilon_0)}]^{N_0}$,平均吸附分子数:$\overline{N}=-\frac{\partial}{\partial\alpha}\ln\Xi=kT\frac{\partial}{\partial\mu}\ln\Xi=\frac{N_0}{1+e^{-\beta(\varepsilon_0+\mu)}}$,达平衡时吸附和未吸附分子的$\mu$,$T$相等,理想气体化学势:$\mu=kT\ln\left[\frac{N}{V}\left(\frac{h^2}{2\pi mkT}\right)^{3/2}\right]$,故吸附率:$\theta=N/N_0=\left[1+\frac{kT}{p}\left(\frac{2\pi mkT}{h^2}\right)^{3/2}e^{-\varepsilon_0/kT}\right]^{-1}$\\
\textbf{近独立粒子平均分布}:系统仅含一种近独立粒子,能级为$\varepsilon_1,\cdots,\varepsilon_l,\cdots$,无简并,当分布为$\{a_l\}$,总粒子数:$N=\sum_la_l$,总能量:$E=\sum_l\varepsilon_la_l$,巨配分函数:$\Xi=\sum_N\sum_se^{-\alpha N-\beta E_s}=\sum_{\{a_l\}}e^{-\sum_l(\alpha+\beta\varepsilon_l)a_l}=\sum_{\{a_l\}}\prod_le^{-(\alpha+\beta\varepsilon_l)a_l}=\prod_l\sum_{\{a_l\}}e^{-(\alpha+\beta\varepsilon_l)a_l}=\prod_l\Xi_l$,$\Xi_l=\sum_{a_l}e^{-(\alpha+\beta\varepsilon_l)a_l}$,能级$\varepsilon_l$上平均粒子数:$\bar{a}_l=\frac{1}{\Xi}\sum_N\sum_sa_le^{-\alpha N-\beta E_s}=\frac{1}{\Xi}\sum_{\{a_m\}}\left[a_le^{-\sum_m(\alpha+\beta\varepsilon_m)a_m}\right]=\frac{1}{\Xi}\sum_{\{a_m\}}\left[a_l\prod_me^{-(\alpha+\beta\varepsilon_m)a_m}\right]=\frac{1}{\Xi}\sum_{\{a_m\}}\left[a_le^{-(\alpha+\beta\varepsilon_l)a_l}\prod_{m\neq l}e^{-(\alpha+\beta\varepsilon_m)a_m}\right]=\frac{1}{\Xi}\left[\sum_{a_l}a_le^{-(\alpha+\beta\varepsilon_l)a_l}\right]\prod_{m\neq l}\left[\sum_{a_m}e^{-(\alpha+\beta\varepsilon_m)a_m}\right]=\frac{1}{\Xi_l}\sum_{a_l}a_le^{-(\alpha+\beta\varepsilon_l)a_l}=\frac{1}{\Xi_l}\left(-\frac{\partial}{\partial\alpha}\right)\Xi_l=-\frac{\partial}{\partial\alpha}\ln\Xi_l$,对玻色子,$a_l$无限制,$\Xi_l=\sum_{a_l=0}^{\infty}e^{-(\alpha+\beta\varepsilon_l)a_l}=\frac{1}{1-e^{-\alpha-\beta\varepsilon_l}}$,$\bar{a}_l=\frac{1}{e^{\alpha+\beta\varepsilon_l}-1}$,对费米子,$a_l=0/1$,$\Xi_l=1+e^{-(\alpha+\beta\varepsilon_l)}$,$a_l=\frac{1}{e^{\alpha+\beta\varepsilon_l}+1}$;若$\omega_l$重简并,$\bar{a}_l=\frac{\omega_l}{e^{\alpha+\beta\varepsilon_l}\pm1}$\\
\textbf{玻色/费米分布的涨落}:视能级$\varepsilon_l$上的粒子为一开系,$\overline{(a_l-\bar{a}_l)^2}=-\frac{\partial\bar{a}_l}{\partial\alpha}=\bar{a}_l(1\pm\bar{a}_l)$;对玻色气体,各能级上粒子数无限制,故涨落较大;对费米气体,$\varepsilon<\mu$的能级上$\bar{a}_l/\omega_l\approx1$,$\varepsilon>\mu$的能级上$\bar{a}_l/\omega_l\approx0$,故涨落很小\\
\textbf{不同能级$\varepsilon_l$,$\varepsilon_m$上玻色/费米分布涨落的关联}:$\overline{(a_l-\bar{a}_l)(a_m-\bar{a}_m)}=\overline{a_la_m}-\bar{a}_l\bar{a}_m$,因$\overline{a_la_m}=\frac{1}{\Xi}\sum_N\sum_sa_la_me^{-\alpha N-\beta E_s}=\frac{1}{\Xi}\sum_{\{a_l\}}a_la_me^{-\sum_k(\alpha+\beta\varepsilon_k)a_k}=\frac{1}{\Xi}\left[\sum_{a_l}a_le^{-(\alpha+\beta\varepsilon_l)a_l}\right]\\\left[\sum_{a_m}a_me^{-(\alpha+\beta\varepsilon_m)a_m}\right]\left[\prod_{k\neq l,m}\sum_{a_k}e^{-(\alpha+\beta\varepsilon_k)a_k}\right]=\frac{1}{\Xi_l}\frac{1}{\Xi_m}\left[\sum_{a_l}a_le^{-(\alpha+\beta\varepsilon_l)a_l}\right]\\\left[\sum_{a_m}a_me^{-(\alpha+\beta\varepsilon_m)a_m}\right]=\bar{a}_l\bar{a}_m$,故$\overline{(a_l-\bar{a}_l)(a_m-\bar{a}_m)}=0$,不同能级上涨落不相关
\end{multicols}

\begin{multicols}{2}
\noindent\textbf{Chap10 涨落理论}
\textbf{10.1 涨落的准热力学理论}
\textbf{涨落}分$2$类:\textbf{宏观量围绕平均值的涨落}(:宏观量瞬时值与平均值的偏差)和布朗运动\\
设系统$E$,$V$,$S$各有平衡值$\overline{E}$,$\overline{V}$,$\overline{S}$,若某微观态有$\Delta E=E-\overline{E}$,$\Delta V=V-\overline{V}$,$\Delta S=S-\overline{S}$,该微观态出现概率为$W(\Delta S,\Delta E,\Delta V)=W_me^{\frac{T\Delta S-\Delta E-p\Delta V}{kT}}$(\textbf{基本公式I})或$W(\Delta S,\Delta E,\Delta V)=W_me^{\frac{\Delta p\Delta V-\Delta T\Delta S}{2kT}}$(\textbf{基本公式II});证明:玻尔兹曼关系给出平衡态的熵$\overline{S}$和系统微观态数极大值$\Omega_m$间的关系:$\overline{S}=k\ln\Omega_m$,由等概率原理,出现$\overline{S}$的概率:$W_m\propto\Omega_m=e^{\overline{S}/k}$,出现$S$的概率:$W\propto\Omega=e^{S/k}$,故孤立系统熵偏差$\Delta S=S-\overline{S}$的概率:$W(\Delta S)=W_me^{\Delta S/k}$,设系统与一大热源接触达热平衡,两者构成的复合系统为孤立系统,有确定$E$和$V$,$\Delta S_0=\Delta S+\Delta S_r\Rightarrow W(\Delta S_0)=W_me^{(\Delta S+\Delta S_r)/k}$,热源很大,平衡时系统温度和压强等于热源温度$T$和压强$p$,由热力学基本方程得$\Delta S_r=\frac{\Delta E_r+p\Delta V_r}{T}=-\frac{\Delta E+p\Delta V}{T}$,代入前式得基公I;以$S$,$V$为自变量,能量偏差$\Delta E=E(S,V)-\overline{E}(\overline{S},\overline{V})$在$(\overline{S},\overline{V})$展开并保留到二阶项得$\Delta E=\left(\frac{\partial E}{\partial S}\right)_V\Delta S+\left(\frac{\partial E}{\partial V}\right)_S\Delta V+\frac{1}{2}\left[\left(\frac{\partial^2E}{\partial S^2}\right)_V(\Delta S)^2+2\frac{\partial^2E}{\partial S\partial V}\Delta S\Delta V+\left(\frac{\partial^2E}{\partial V^2}\right)_S(\Delta V)^2\right]$,其中各级偏导取$S=\overline{S}$,$V=\overline{V}$时的值,代入$\left(\frac{\partial E}{\partial S}\right)_V=T$,$\left(\frac{\partial E}{\partial V}\right)_S=-p$得$\Delta E=T\Delta S-p\Delta V+\frac{1}{2}(\Delta T\Delta S-\Delta p\Delta V)$,将上式代入I得基公II\\
\textbf{基公的应用}:基公II中4个偏差仅2个独立,可选$2$个变量$X$,$Y$作自变量,利用基公II求$\overline{(\Delta X)^2}$,$\overline{(\Delta Y)^2}$,$\overline{\Delta X\Delta Y}$\\
\textbf{以$T$,$V$为自变量},$\Delta S=\left(\frac{\partial S}{\partial T}\right)_V\Delta T+\left(\frac{\partial S}{\partial V}\right)_T\Delta V=\frac{C_V}{T}\Delta T+\left(\frac{\partial p}{\partial T}\right)_V\Delta V$,$\Delta p=\left(\frac{\partial p}{\partial T}\right)_V\Delta T+\left(\frac{\partial p}{\partial V}\right)_V\Delta V$,代入基公II得$W(\Delta T,\Delta V)=W_m\exp\left[-\frac{C_V}{2kT^2}(\Delta T)^2+\frac{1}{2kT}\left(\frac{\partial}{\partial V}\right)_T(\Delta V)^2\right]$,于是$\overline{(\Delta T)^2}=\frac{\int_{-\infty}^{+\infty}\int_{-\infty}^{+\infty}(\Delta T)^2W(\Delta T,\Delta V)d(\Delta T)d(\Delta V)}{\int_{-\infty}^{+\infty}\int_{-\infty}^{+\infty}W(\Delta T,\Delta V)d(\Delta T)d(\Delta V)}=\frac{\int_{-\infty}^{+\infty}(\Delta T)^2\exp\left[-\frac{C_V}{2kT^2}(\Delta T)^2\right]d(\Delta T)}{\int_{-\infty}^{+\infty}\exp\left[-\frac{C_V}{2kT^2}(\Delta T)\right]}=\frac{kT^2}{C_V}$,$\overline{(\Delta V)^2}=\frac{\int_{-\infty}^{+\infty}\int_{-\infty}^{+\infty}(\Delta V)^2W(\Delta T,\Delta V)d(\Delta T)d(\Delta V)}{\int_{-\infty}^{+\infty}\int_{-\infty}^{+\infty}W(\Delta T,\Delta V)d(\Delta T)d(\Delta V)}=\frac{\int_{-\infty}^{+\infty}(\Delta V)^2\exp\left[\frac{1}{2kT}\left(\frac{\partial p}{\partial V}\right)_T(\Delta V)^2\right]d(\Delta V)}{\int_{-\infty}^{+\infty}\exp\left[\frac{1}{2kT}\left(\frac{\partial p}{\partial V}\right)_T(\Delta V)^2\right]d(\Delta V)}=-kT\left(\frac{\partial V}{\partial p}\right)_T=kTV\kappa_T$,$\overline{\Delta T\Delta V}=\overline{\Delta T}~\overline{\Delta V}=0$,这说明$T$和$V$统计独立\\
\textbf{以$S$,$p$为自变量},$W(\Delta S,\Delta p)=W_m\exp\left[-\frac{1}{2kC_p}(\Delta S)^2+\frac{1}{2kT}\left(\frac{\partial V}{\partial p}\right)_S(\Delta p)^2\right]$,于是$\overline{(\Delta S)^2}=kC_p$,$\overline{(\Delta p)^2}=-kT\left(\frac{\partial p}{\partial V}\right)_S$,$\overline{\Delta S\Delta p}=\overline{\Delta S}\overline{\Delta p}=0$\\
\textbf{其他相关函数}:以$T$,$V$为自变量,$\Delta T\Delta S=\left(\frac{\partial S}{\partial T}\right)_V(\Delta T)^2+\left(\frac{\partial S}{\partial V}\right)_T\Delta T\Delta V$,$\overline{\Delta T\Delta S}=\left(\frac{\partial S}{\partial T}\right)_V\overline{(\Delta T)^2}+\left(\frac{\partial S}{\partial V}\right)_T\overline{\Delta T\Delta V}=\frac{C_V}{T}\overline{(\Delta T)^2}=kT$,$\overline{\Delta V\Delta p}=-kT$,$\overline{\Delta S\Delta V}=kT\left(\frac{\partial V}{\partial T}\right)_p$,$\overline{\Delta T\Delta p}=\frac{kT}{C_V}\left(\frac{\partial p}{\partial T}\right)_V$\\
\textbf{粒子数$N$的涨落}:当$N$固定,$nV=N\Rightarrow\frac{\Delta n}{n}+\frac{\Delta V}{V}=0$,\textbf{粒子数密度相对涨落}:$\frac{\overline{(\Delta n)^2}}{n^2}=\frac{\overline{(\Delta V)^2}}{V^2}=\frac{kT}{V}\kappa_T$;当$V$固定,$\frac{\overline{(\Delta N)^2}}{N^2}=\frac{\overline{(\Delta n)^2}}{n^2}=\frac{1}{V}kT\kappa_T$\\
\textbf{能量$E$的涨落}:以$T$,$V$为自变量,$\Delta E=\left(\frac{\partial E}{\partial T}\right)_V\Delta T+\left(\frac{\partial E}{\partial V}\right)_T\Delta V=C_V\Delta T+\left(\frac{\partial E}{\partial V}\right)_T\Delta V\Rightarrow\overline{(\Delta E)^2}=C_V^2\overline{(\Delta T)^2}+2C_V\left(\frac{\partial E}{\partial V}\right)_T\overline{\Delta T\Delta V}+\left(\frac{\partial E}{\partial V}\right)_T\overline{(\Delta V)^2}=kT^2C_V+kTV\kappa_T\left(\frac{\partial E}{\partial V}\right)_T^2$,由$N$不变换到$V$不变,因$\left(\frac{\partial E}{\partial V}\right)_T=\frac{N}{V}\left(\frac{\partial E}{\partial N}\right)_T$得$\overline{(\Delta E)^2}=kT^2C_V+\frac{N^2}{V}kT\kappa_T\left(\frac{\partial E}{\partial N}\right)_T^2=kT^2C_V+\overline{(\Delta N)^2}\left(\frac{\partial E}{\partial N}\right)_T$\\
\textbf{10.5 布朗运动}:处于气或液体中的微小颗粒因受周围气或液体分子碰撞而产生的不规则随机运动;布朗粒子通常很小(直径$10^{-7}\sim10^{-6}$m);粒子越小,布朗运动越显著\\
\textbf{郎之万理论}:一维颗粒运动方程:$m\frac{d^2x}{dt^2}=f(t)+g(t)$,($g(t)$--可能存在的其它作用力,如电磁力/重力,$f(t)=-\alpha v+F(t)$--介质分子施于颗粒的净作用力($-\alpha v$--粘滞阻力,由\textbf{斯托克斯公式},$\alpha=6\pi a\eta$,$a$--颗粒半径,$\eta$--粘滞系数,$F(t)$--涨落力,相当于分子对静止布朗颗粒的碰撞作用力,平均值$=0$)),从而$m\frac{d^2x}{dt^2}=-\alpha\frac{dx}{dt}+F(t)+G(t)$(\textbf{郎之万方程}),当无其它外力,$m\frac{d^2x}{dt^2}=-\alpha\frac{dx}{dt}+F(t)$,两边同乘$x$并用$x\ddot{x}=\frac{d}{dt}(x\dot{x})-\dot{x}^2=\frac{1}{2}\frac{d^2}{dt^2}x^2-\dot{x}^2$得$\frac{1}{2}\frac{d^2}{dt^2}(mx^2)-m\dot{x}^2=-\frac{\alpha}{2}\frac{d}{dt}x^2+xF(x)$,因(1)求平均与对时间求导可交换:$\overline{\frac{d}{dt}x^2}=\frac{d}{dx}\overline{x^2}$,$\overline{\frac{d}{dt}mx^2}=\frac{d}{dt}\overline{mx^2}$,(2)涨落力与颗粒位置无关:$\overline{xF(t)}=\bar{x}\overline{F(t)}=\bar{x}\cdot0=0$,(3)当颗粒与介质达热平衡,由均分定理,颗粒平均动能:$\frac{1}{2}\overline{m\dot{x}^2}=\frac{1}{2}kT$,得$\frac{d^2}{dt^2}\overline{x^2}-\frac{\alpha}{m}\frac{d}{dt}\overline{x^2}-\frac{2kT}{m}=0$,方程通解为$\overline{x^2}=\frac{2kT}{\alpha}t+C_1e^{-\frac{\alpha}{m}t}+C_2$,设$m=\frac{4}{3}\pi\rho a^3$,则$\frac{\alpha}{m}=\frac{9\eta}{2a^2\rho}$,因$\frac{\alpha}{m}\sim10^7s^{-1}$,故很短时间($>10^{-6}$s)后,通解中第二项便可忽略,设$x(0)=0$得$C_2=0$,从而得$\overline{x^2}=\frac{2kT}{\alpha}t$(\textbf{爱因斯坦公式})\\
\textbf{从扩散观点看布朗运动}:一维情况下,\textbf{菲克定律}:$\bm{J}=-D\nabla n$,\textbf{连续方程}:$\frac{\partial n}{\partial t}+\nabla\cdot\bm{J}=0$,其中$n(x,t)$--布朗颗粒密度分布,$\bm{J}(x,t)$--布朗颗粒流量(单位时间通过单位截面的颗粒数),$\Rightarrow\frac{\partial n}{\partial t}=D\nabla^2n$,设$x(0)=0$即$n(x,0)=N\delta(x)$,则$n(x,t)=\frac{N}{2\sqrt{\pi Dt}}e^{-\frac{x^2}{4Dt}}$,这说明颗粒密度分布为与$t$有关的高斯分布,$\overline{x^2}=\int x^2\rho(x)dx=\int x^2\frac{n(x,t)}{N}dx=\frac{1}{\sqrt{\pi Dt}}\int_{-\infty}^{+\infty}x^2e^{-\frac{x^2}{4Dt}}dx=2Dt$,与爱因斯坦方程形式一致,比较得$D=\frac{kT}{\alpha}=\frac{kT}{6\pi\alpha\eta}$\\
\textbf{10.6 布朗颗粒动量的扩散和时间关联}
\textbf{系综平均}:某物理量对大量布朗粒子的平均$\overline{A(t)}=\frac{1}{N}\sum_{i=1}^{N}A_i(t)$,其中$i$--第$i$个粒子\\
\textbf{涨落力的时间关联函数}:$\overline{F(t)F(t+\tau)}=\frac{1}{N}\sum_{i=1}^NF_i(t)F_i(t+\tau)$,当$\tau$足够长,$F_i(t)$和$F_i(t+\tau)$互不关联,$\overline{F(t)F(t+\tau)}=0$,反之相互依赖,由此引入\textbf{涨落力的关联时间}$\tau_c$区分两种情况;\textbf{关联时间尺度}上:关联时间与涨落力的平均周期一般有相同数量级,对在$\tau_c$量级时间内仅有微小变化的物理量,$\overline{F(t)F(t+\tau)}=2D_p\delta(\tau)$,其中$2D_p$--涨落力强度的量度,这说明不同时刻涨落力不关联;\textbf{长时间平均值}:$\langle F(t)F(t+\tau)\rangle=\lim_{T_0\rightarrow\infty}\frac{1}{T_0}\int_0^{T_0}F(t)F(t+\tau)dt$,长时间内,颗粒将经历各种可能的涨落力作用,故长时间均值与系综平均相等,$\langle F(t)F(t+\tau)\rangle=\overline{F(t)F(t+\tau)}$\\
\textbf{动量平均值和散差}:无外力时郎之万方程为$\frac{dp}{dt}=-\gamma p+F(t)$,其中$\gamma=\alpha/m$,$\Rightarrow\frac{d}{dt}(pe^{-\gamma t})=e^{\gamma t}F(t)\Rightarrow p(t)=p(0)e^{-\gamma t}+e^{-\gamma t}\int_0^te^{-\gamma\xi}F(\xi)d\xi$,因$\overline{F}(\xi)=0$得\textbf{动量平均值}:$\overline{p(t)}=p(0)e^{-\gamma t}$,\textbf{动量散差}:$\overline{(\Delta p)^2}=\overline{[p(t)-\overline{p(t)}]^2}=\int_0^td\xi'\int_0^t\overline{F(\xi)F(\xi')}e^{-\gamma(t-\xi)}e^{-\gamma(t-\xi')}d\xi=2D_p\delta(\tau)\int_0^te^{-2\gamma(t-\xi)}d\xi=\frac{D_p}{\gamma}[1-e^{-2\gamma t}]$;当$\tau_c\ll t\ll1/\gamma$,$\overline{(\Delta p)^2}=2D_pt$,动量散差与位移平方平均值有类似规律,$D_p$--动量扩散系数;当$t\gg1/\gamma$,$p(t)=0$,$\overline{(\Delta p)^2}=\overline{p^2}=\frac{D_p}{\gamma}$,$\frac{\overline{p^2}}{2m}=\frac{1}{2}kT\Rightarrow D_p=m\gamma kT=\alpha kT$,阻尼系数$\propto$动量扩散系数,粘滞阻力导致颗粒动能耗散和颗粒与介质达平衡,涨落力导致动量扩散;当$t>1/\gamma$,$p(t)\approx e^{-\gamma t}\int_0^te^{\gamma\xi}F(\xi)d\xi$,$\overline{p(t)p(t')}=e^{-\gamma(t+t')}\int_0^td\xi\int_0^{t'}d\xi'\overline{F(\xi)F(\xi')}e^{\gamma(\xi+\xi')}=2D_p\int_0^td\xi\int_0^{t'}d\xi'\delta(\xi-\xi')e^{-\gamma(t-\xi)}e^{-\gamma(t'-\xi')}$,$\overline{p(t)p(t')}=\frac{D_p}{\gamma}[e^{-\gamma(t-t')}-e^{-\gamma(t+t')}],(t>t')$,$\overline{p(t)p(t')}=\frac{D_p}{\gamma}[e^{-\gamma(t'-t)}-e^{-\gamma(t'+t)}],(t<t')$,$\overline{p(t)p(t')}=\frac{D_p}{\gamma}[e^{-\gamma|t'-t|}-e^{-\gamma|t'+t|}]\approx\frac{D_p}{\gamma}e^{-\gamma|t'-t|}=mkTe^{-\gamma|t'-t|},(t,t'>1/\gamma)$,不同时刻的涨落力无关联,不同时刻的动量却相互关联,因为动量是平均力和涨落力共同作用的结果,是积分效应\\
\textbf{布朗颗粒的位移}:经时间$t$后,\textbf{位移}:$x(t)=\frac{1}{m}\int_0^tp(\xi)d\xi$,\textbf{位移平方平均值}:$\overline{x^2(t)}=\frac{1}{m^2}\int_0^td\xi\int_0^td\xi'\overline{p(\xi)p(\xi')}=\frac{kT}{m}\int_0^td\xi\int_0^td\xi'e^{-\gamma|\xi-\xi'|}=\frac{2kT}{m\gamma}t=\frac{2kT}{\alpha}t$,与布朗理论结果相同
\end{multicols}

\begin{multicols}{2}
\noindent\textbf{附录C 统计物理学常用的积分公式}
$\int_0^{\infty}e^{-x^2}dx=\frac{\sqrt{\pi}}{2}$\quad\quad
$\Gamma(n)=\int_0^{\infty}e^{-x}x^{n-1}dx=(n-1)!$,$\Gamma(n+\frac{1}{2})=(n-\frac{1}{2})(n-\frac{3}{2})\cdots\frac{1}{2}\sqrt{\pi}$\quad\quad
$I(n)=\int_0^{\infty}e^{-\alpha x^2}x^ndx$,$I(0)=\frac{\sqrt{\pi}}{2\alpha^{1/2}}$,$I(1)=\frac{1}{2\alpha}$,$I(2)=\frac{\sqrt{\pi}}{4\alpha^{3/2}}$,$I(3)=\frac{1}{2\alpha^2}$,$I(4)=\frac{3\sqrt{\pi}}{8\alpha^{5/2}}$,$I(5)=\frac{1}{\alpha^3}$\\
$I(n)=\int_0^{\infty}\frac{x^{n-1}}{e^x-1}dx$,$I(2)=\frac{\pi^2}{6}$,$I(3)=2\sum_{k=1}^{\infty}\frac{1}{k^3}\approx2.404$,$I(4)=\frac{\pi^4}{15}$,$I(\frac{3}{2})=\frac{\sqrt{\pi}}{2}\sum_{k=1}^{\infty}\frac{1}{k^{3/2}}=\frac{\sqrt{\pi}}{2}\times2.612$,$I(\frac{5}{2})=\frac{3\sqrt{\pi}}{4}\sum_{k=1}^{\infty}\frac{1}{k^{5/2}}=\frac{3\sqrt{\pi}}{4}\times1.341$\quad\quad
$\int_0^{\infty}\frac{xdx}{e^x+1}=\frac{\pi^2}{12}$
\end{multicols}
\end{document}
